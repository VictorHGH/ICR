\author{González Hernández Victor Hugo}
\title{Idonea comunicación de resultados}

% ==== Tipografía e idioma (XeLaTeX) ====
\usepackage{booktabs}
\usepackage{fontspec}
\setmainfont{Liberation Sans}

\usepackage[spanish]{babel}
\usepackage[style=english,autostyle=false]{csquotes} % “comillas inglesas” con \enquote

% ==== Microtipografía y ajustes anti-overfull ====
\usepackage{microtype}
\tolerance=3000
\emergencystretch=3em
\hfuzz=0.5pt
\microtypesetup{protrusion=true}

% ==== Maquetación y utilidades ====
\usepackage[left=2.54cm, right=2.54cm, top=2.54cm, bottom=2.54cm]{geometry}
\usepackage{titlesec}
\usepackage{titling}
\usepackage{enumitem}
\usepackage{graphicx}
\usepackage{xcolor}
\usepackage{fancyhdr}
\usepackage{pgfgantt}
\usepackage{setspace}
\setstretch{1.5}

% TikZ (necesario para \usetikzlibrary)
\usepackage{tikz}
\usetikzlibrary{mindmap}

% ==== Bibliografía y enlaces ====
\usepackage{natbib}
\bibliographystyle{agsm}
\setcitestyle{yysep={,}}
\usepackage[hidelinks]{hyperref}

% ==== Guionado manual de palabras difíciles ====
\hyphenation{
  Mez-qui-tal
  bio-cul-tu-ral
  ver-ná-cu-la
  ver-ná-cu-las
  au-to-cons-cien-te
  le-chu-gui-lla
  qui-o-tes
  hñähñu
  Wä-da
  sem-i-de-sér-ti-co
  pa-tri-mo-nio
}

% ==== Títulos ====
\titleformat{\section}
{\large\bfseries}
{\thesection}
{0.5em}
{}

\titleformat{\subsection}
{\normalsize\bfseries}
{\thesubsection}
{0.5em}
{}

\titleformat{\subsubsection}
{\normalsize\bfseries}
{\thesubsubsection}
{0.5em}
{}

\titlespacing*{\section}{0pt}{0.5em}{0.5em}
\titlespacing*{\subsection}{0pt}{0.5em}{0.5em}
\titlespacing*{\subsubsection}{0pt}{0.5em}{0.5em}

% ==== Listas ====
\setlist{itemsep=1.5pt, parsep=0pt, topsep=0pt, partopsep=0pt}

% ==== Encabezados y pies ====
\setlength{\headheight}{20.6pt}
\fancyhf{}
\lhead[\leftmark]{\textcolor{gray}{UAM-Xochimilco}}
\rhead[\textcolor{gray}{UAM-Xochimilco}]{}
\lfoot[\thepage]{\textcolor{gray}{González H. Victor H.}}
\rfoot[\textcolor{gray}{González H. Victor H.}]{\thepage}
\renewcommand{\headrulewidth}{0.3pt}
\renewcommand{\footrulewidth}{0.3pt}
\pagestyle{fancy}

% ==== Traducciones ====
\addto\captionsspanish{\renewcommand{\contentsname}{Índice}}

