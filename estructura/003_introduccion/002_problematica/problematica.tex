\subsection{Problemática}

La arquitectura vernácula del Valle del Mezquital constituye una expresión viva del conocimiento tradicional y de la relación simbiótica entre los habitantes y su entorno natural. Sin embargo, estos sistemas constructivos han experimentado un proceso acelerado de abandono y transformación, derivado de múltiples factores como la migración, la industrialización, los cambios socioeconómicos, la transculturación ideológica y la falta de reconocimiento institucional hacia su valor cultural.

A pesar de ello, aún es posible identificar algunas viviendas tradicionales que conservan parte de sus formas y materiales originales. No obstante, estas estructuras, descritas por Rudofsky como ``arquitectura sin pedigrí, arquitectura no formal, arquitectura no clasificada'' \citep[p. 9]{rudofsky1964}, han sido progresivamente modificadas. Con frecuencia, parte de su estructura, acabados o configuración espacial son sustituidos por elementos industrializados, lo cual transforma no solo su apariencia sino también la forma en que se habitan y perciben estos espacios.

Si bien fenómenos como la migración laboral, relacionada con las desigualdades rurales-urbanas y la precarización del campo, han tenido un impacto considerable \citep{monroy2009, boils2010dadho}, no son la única causa de esta transformación. La desvalorización simbólica de las construcciones vernáculas, históricamente relegadas a un estatus inferior dentro de la arquitectura \citep[p. 2]{torres1999revista}, y la ausencia de políticas públicas que reconozcan y fomenten su preservación, han sido factores igualmente determinantes.

Entre las opiniones recogidas en la comunidad, se percibe una tendencia creciente a valorar los materiales industrializados por encima de los tradicionales, al asociarlos con mayor resistencia, estatus y durabilidad. Asimismo, las personas mayores expresan su preocupación ante la falta de interés de las nuevas generaciones por aprender y continuar las prácticas constructivas tradicionales, pues gran parte de su tiempo y energía está enfocada en generar ingresos económicos.

Este desplazamiento de los saberes constructivos tradicionales no implica únicamente la pérdida material de edificaciones, sino también la desaparición de prácticas culturales, conocimientos ecológicos locales y formas sostenibles de habitar que constituían un componente esencial de la identidad colectiva de estas comunidades.

Ante este escenario, resulta fundamental analizar las dinámicas que han llevado a la desvalorización de los sistemas constructivos vernáculos, con el propósito de proponer estrategias que fomenten su reconocimiento, revitalización y preservación como parte integral del patrimonio cultural del Valle del Mezquital.
