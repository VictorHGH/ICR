\subsection{Contexto general}

El Valle del Mezquital, ubicado en el estado de Hidalgo, fue inicialmente habitado por grupos otomíes. Tras la conquista española hacia 1550, la región ha preservado una rica y diversa expresión cultural, expresada en danzas, música, celebraciones religiosas y vestimentas que reflejan la profunda conexión entre sus habitantes y el entorno natural, así como su capacidad de adaptación a las transformaciones históricas.

Una manifestación de esta conexión es la arquitectura vernácula del Valle del Mezquital, en la cual persiste el uso de materiales locales como pencas de maguey, tepetate, quiotes de lechuguilla y maguey, horcones de madera, tierra compactada y, en algunos casos, cercos vivos de cactus ``órgano'' que cumplen funciones de resguardo y delimitación de espacios habitables.

Desde una perspectiva conceptual, las construcciones vernáculas pueden ser entendidas dentro del patrimonio cultural, definido como "todo aquello que caracteriza al hombre como especie autoconsciente que genera mediante dichas manifestaciones, cultura" \citep{dominguez2004pautas}, y como ``todo aquello que socialmente se considera digno de conservación, independientemente de su interés utilitario'' \citep[p. 63]{prats1998concepto}. Esta visión implica que el patrimonio, natural, cultural o biocultural, es una construcción social dinámica, sujeta a cambios en función de intereses colectivos.

Sin embargo, en el contexto contemporáneo, el patrimonio enfrenta riesgos de desnaturalización, como lo advierte \cite{i2008zombi} al referirse al ``zombie de la modernidad'' como manifestaciones revividas artificialmente para cumplir funciones actuales como la legitimación de identidades, la promoción turística o la generación de productos de consumo. Esta instrumentalización pone en peligro la autenticidad de las tradiciones y la memoria colectiva de las comunidades.

Un caso ejemplar de resistencia cultural es el grupo Wäda, colectivo de artesanos pertenecientes a la comunidad de El Deca, municipio de Cardonal, cuya misión expresa lo siguiente: ``preservar sus actividades productivas tradicionales mediante la elaboración y comercialización de productos de fibra de maguey y bordado de textiles, así mismo busca incidir en las nuevas generaciones mediante la transmisión de sus conocimientos''\footnote{Cita tomada de un cartel expuesto en el taller Wäda.}

De igual manera, el grupo Wäda ha mostrado su interés en recobrar sus prácticas de construcción vernácula, con el fin de preservar sus conocimientos tradicionales, fomentando el uso de estos en proyectos que también ayuden a incrementar su economía, turismo y valor a través de su legado cultural.
