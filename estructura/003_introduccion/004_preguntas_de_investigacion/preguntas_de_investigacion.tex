\subsection{Preguntas de investigación}

\begin{enumerate}
	\item {¿Cómo se ha desnaturalizado, de 1990 a 2024, la tradición constructiva de la vivienda vernácula hñähñu del Valle del Mezquital, y qué valores patrimoniales históricos y identitarios se ven más afectados?}

	\item {¿Cuáles son los principales cambios constructivos y simbólicos derivados de esa pérdida de saberes, y qué actores o procesos los impulsan?}

	\item {¿En qué medida la normativa patrimonial vigente (Cartas ICOMOS, Ley 2023) reconoce la dimensión inmaterial de la técnica tradicional, y dónde se generan tensiones entre esos marcos y las prácticas vivas de la comunidad?}

	\item {¿Cómo inciden los costos de mantenimiento, la disponibilidad de mano de obra experta y la oferta de materiales industrializados en las decisiones locales de conservar, adaptar o sustituir las construcciones tradicionales?}

	\item {¿Qué estrategias participativas (técnicas, educativas y normativas) pueden revitalizar la transmisión del saber constructivo hñähñu, garantizando habitabilidad contemporánea y salvaguarda del patrimonio biocultural?}
\end{enumerate}
