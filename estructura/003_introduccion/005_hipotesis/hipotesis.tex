\subsection{Hipótesis}

La pérdida de continuidad intergeneracional del saber constructivo hñähñu no es un efecto mecánico de la escasez material, sino el resultado de la desarticulación de los mecanismos culturales de transmisión (faenas comunales, aprendizaje maestro-oficial, rituales domésticos y oralidad técnica).  Cuando estos mecanismos se debilitan, la vivienda vernácula pierde su valor identitario y simbólico para la comunidad, lo que favorece su reemplazo por tipologías industrializadas, aun en entornos donde persisten condiciones ambientales o normativas favorables a la conservación.
