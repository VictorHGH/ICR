\subsection{Hipótesis}

La pérdida de continuidad intergeneracional del saber constructivo hñähñu no es un efecto de la escasez material, sino el resultado de la desarticulación de los mecanismos culturales de transmisión (faenas comunales, aprendizaje maestro-oficial, rituales domésticos y oralidad técnica).  Cuando estos mecanismos se debilitan, la vivienda vernácula pierde su valor identitario y simbólico para la comunidad, lo que favorece su reemplazo por tipologías industrializadas, modificaciones en los métodos o transformaciones que provocan el olvido y perdida un apartado importante en la cultura hñähñu.

Hoy en día, aun persisten viviendas tradicionales hñähñu, en las zonas rurales en las comunidades de hidalgo

Los métodos constructivos tradicionales que persisten en las comunidades rurales hñähñu del Valle del Mezquital, Hidalgo son parte de la cultura tradicional de México, que paulatinamente esta desapareciendo debido a cambios generacionales, perdida de identidad, falta de transmisión de conocimientos
