\subsection{Hipótesis}

% La pérdida de continuidad intergeneracional del saber constructivo hñähñu no es un efecto de la escasez material, sino el resultado de la desarticulación de los mecanismos culturales de transmisión (faenas comunales, aprendizaje maestro-oficial, rituales domésticos y oralidad técnica).  Cuando estos mecanismos se debilitan, la vivienda vernácula pierde su valor identitario y simbólico para la comunidad, lo que favorece su reemplazo por tipologías industrializadas, modificaciones en los métodos o transformaciones que provocan el olvido y perdida de un apartado importante en la cultura hñähñu.

% Hoy en día, aun persisten viviendas tradicionales hñähñu, en las zonas rurales en las comunidades de hidalgo

% Los métodos constructivos tradicionales que persisten en las comunidades rurales hñähñu del Valle del Mezquital, Hidalgo son parte de la cultura tradicional de México, que paulatinamente esta desapareciendo debido a cambios generacionales, perdida de identidad, falta de transmisión de conocimientos

La pérdida de la transmisión entre generaciones del saber constructivo vernáculo hñähñu no responde principalmente a factores de escasez material, costos o disponibilidad de mano de obra, sino al debilitamiento de los mecanismos culturales de transmisión (faenas comunales, aprendizaje maestro-oficial, rituales domésticos y oralidad técnica). Al interrumpirse estos procesos, la vivienda vernácula pierde su valor identitario y simbólico para la comunidad, lo que favorece tanto su sustitución por tipologías industrializadas como la modificación de sus métodos y la transformación de sus significados patrimoniales.

% La pérdida de la transmisión entre generaciones del saber constructivo hñähñu se explica principalmente por la desarticulación de los mecanismos culturales de transmisión (faenas comunales, aprendizaje maestro-oficial, rituales domésticos y oralidad técnica). Este debilitamiento, más que la sola escasez material o la disponibilidad de mano de obra, ha erosionado el valor identitario y simbólico de la vivienda vernácula, favoreciendo su sustitución por tipologías industrializadas y la alteración de sus métodos constructivos. Factores económicos y la influencia de políticas patrimoniales institucionales han intensificado este proceso, pero no constituyen su causa principal.
