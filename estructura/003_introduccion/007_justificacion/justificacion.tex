\section{Justificación}
% Dichas transformaciones no solo han afectado las formas de habitar, arquitectónicas y sociales, sino también han afectado al paisaje, cultura y tradiciones, por lo que es importante tener documentado dicho proceso, tanto social, constructivo, y teniendo en cuenta que. Sin embargo esta trasferencia de conocimiento se ha visto afectada debido que hoy en día los jóvenes han ido perdiendo el interés por retomar las tradiciones y costumbres de sus comunidades natales. Es por esto que la documentación, análisis y difusión de las trasformaciones arquitectónicas no solo buscan resguardar los conocimientos, sino entender el por que de estos cambios y la resignificación que la vivienda tradicional ha tenido en los últimos años tomando como referencia los datos estadísticos propuestos tanto por el Instituto Nacional de Estadística y Geografía (INEGI) así como estudios relacionados tanto con la arquitectura vernácula y el patrimonio cultural que esta representa.

Frente a este panorama, resulta necesario investigar de manera integral los factores que han conducido a la desvalorización de la vivienda vernácula en el Valle del Mezquital. A diferencia de enfoques que separan lo material de lo simbólico, esta investigación propone abordar la vivienda como una totalidad cultural, ecológica y social. El objetivo es aportar al reconocimiento de estos sistemas constructivos como parte del patrimonio biocultural de la región, integrando las voces de las comunidades, sus saberes y sus formas de vida como base para su preservación y resignificación.
