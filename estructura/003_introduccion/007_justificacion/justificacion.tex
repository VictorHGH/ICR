\subsection{Justificación}

La vivienda vernácula hñähñu del Valle del Mezquital representa, desde
una perspectiva biocultural, la cristalización de prácticas adaptativas
creadas durante siglos en un entorno semidesértico.  Su pertinencia
va mas allá la nostalgia estética: las técnicas de tierra cruda,
aprovechamiento de la penca de maguey y organización comunal de faenas
constituyen respuestas de bajo impacto ambiental que hoy dialogan con
las agendas globales de sostenibilidad y resiliencia \citep{gudynas2010desarrollo}.
Sin embargo, a pesar del reconocimiento doctrinal de la
\emph{Carta del Patrimonio Vernáculo Construido} \citep{icomos1999carta}
y de la reciente \emph{Ley Federal de Protección del Patrimonio Cultural
	de los Pueblos y Comunidades Indígenas y Afromexicanas}
(\citeyear{ley2023patrimonio}), la vivienda tradicional continúa cediendo
frente a materiales industrializados, modelos habitacionales ajenos y
marcos normativos que privilegian estándares universales de ``vivienda
digna'' sobre criterios culturales situados.

\textbf{Vacíos de conocimiento.}
Los diagnósticos existentes se fragmentan entre visiones materiales
(eficiencia térmica, estabilidad estructural) y lecturas simbólicas
(costumbre, identidad) sin articular la doble dimensión
eco-socio-técnica.  Aun menos frecuente es el diálogo directo con los
portadores del saber: la literatura sobre el Mezquital suele citar la
migración y la pobreza como causas genéricas de abandono
\citep{alvarez2003maguey, juarez2018transformacion}, pero carece de
evidencia sobre cómo se interrumpe la cadena
maestro-oficial o por qué ciertos rituales de construcción se
desactivan antes que otros.  Cubrir esos vacíos permitirá precisar el
rol de factores aparentemente secundarios, mercado de materiales,
políticas de crédito, estigmas escolares, que, en conjunto, erosionan la
tradición.

\textbf{Relevancia social y cultural.}
Para las comunidades hñähñu, la casa de adobe no es solo refugio
climático; es nodo de sociabilidad, altar doméstico y archivo de la
memoria familiar.  La pérdida de estas viviendas amenaza la continuidad
de la lengua, la gastronomía basada en hornos de tierra y los sistemas
de reciprocidad que organizan la vida comunal.  Investigar la
desvalorización desde un enfoque participativo responde, por tanto, a
una demanda ética de salvaguarda de derechos culturales colectivos,
consagrados en la legislación nacional de 2023.

\textbf{Contribución académica.}
La propuesta integra tres campos que raramente convergen en los
estudios de arquitectura vernácula mexicana:

\begin{enumerate}
	\item la matriz de valores patrimoniales de De la Torre-Mason
	      (\citeyear{delatorre2002values}), aplicada por primera vez a
	      vivienda indígena;
	\item el concepto de patrimonio \emph{biocultural}, que permite medir
	      simultáneamente vitalidad técnica y biodiversidad asociada;
	\item modelos de transmisión cultural que conectan rupturas
	      intergeneracionales con indicadores cuantificables (Índice de
	      Vitalidad Técnica).
\end{enumerate}

Tal convergencia aportará un marco replicable para otras regiones
donde la conservación se debate entre uso cotidiano y museificación.

\textbf{Impacto potencial.}
Los resultados podrán servir de insumo a:

\begin{itemize}
	\item programas de vivienda estatal interesados en soluciones de bajo
	      carbono compatibles con códigos locales de construcción;
	\item talleres comunitarios de formación de jóvenes albañiles,
	      fortaleciendo la transmisión vertical del saber;
	\item iniciativas de turismo cultural que requieran guías de
	      intervención mínima y relatos interpretativos coherentes con la
	      cosmovisión hñähñu.
\end{itemize}

En síntesis, investigar la erosión de la tradición constructiva hñähñu
no solo llena un hueco académico; propone una vía para reactivar,
desde la participación comunitaria, el valor patrimonial vivo de la
vivienda vernácula y contribuir al equilibrio ecológico y cultural del
Valle del Mezquital.
