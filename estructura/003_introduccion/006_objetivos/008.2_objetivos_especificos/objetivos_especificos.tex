\subsection{Objetivos Específicos}
%``Los objetivos específicos nacen del objetivo general y tienen que cumplir con las mismas pautas``

%\subsubsection{partes del 1er Objetivo}

%\textbf{¿Qué?} = Documentar los sistemas constructivos tradicionales mas representativos de las comunidades,

%\textbf{¿Cómo?} = mediante levantamientos arquitectónicos, fotografías, videos y modelos 3D,

%\textbf{¿Para?} = para permitir tener un mejor entendimiento de sus partes, dimensiones y materiales utilizados.

%\subsubsection{Primer objetivo completo}

%Documentar los sistemas constructivos tradicionales mas representativos de las comunidades, mediante levantamientos arquitectónicos, planos, fotografías y modelos 3D, para permitir tener un mejor entendimiento de sus partes, dimensiones y materiales utilizados.

%\subsubsection{Partes del 2do Objetivo}

%\textbf{¿Qué?} = Realizar entrevistas con los habitantes de las viviendas tradicionales

%\textbf{¿Cómo?} = mediante visitas de campo,

%\textbf{¿Para?} = para poder profundizar en el sentir de sus usuarios y sus ideales hacia este tipo de construcciones.

%\subsubsection{Segundo objetivo completo}

%Realizar entrevistas con los habitantes de las viviendas tradicionales mediante visitas de campo, para poder profundizar en el sentir de sus usuarios y sus ideales hacia este tipo de construcciones.

%\subsubsection{Partes del 3er Objetivo}

%\textbf{¿Qué?} = Experimentar con los materiales constructivos,

%\textbf{¿Cómo?} = mediante pruebas de tiempo y laboratorio

%\textbf{¿Para?} = poder identificar relaciones entre la cosmovisión referente a la recolección de los materiales y el deterioro de los materiales.

%\subsubsection{Tercer objetivo completo}

%Experimentar con los materiales constructivos, mediante pruebas de tiempo y laboratorio para poder identificar relaciones entre la cosmovisión referente a la recolección de los materiales y el deterioro de los materiales.

\begin{enumerate}

  \item{Documentar histórica y diacronicamente la vivienda tradicional para contar con un registro de los cambios arquitectónicos transcurridos entre los años 1990-2022, mediante levantamientos arquitectónicos, aplicación de entrevistas a profundidad y abiertas a los habitantes, fotografías.} 

  \item{Analizar las trasformaciones arquitectónicas detectadas más relevantes, producto de la incidencia de la migración, mediante la comparativa de distribución espacial y formas de habitar, así como uso de materiales para la identificación de patrones de cambio y permanencia.}

  \item{Determinar la significación actual de la vivienda tradicional, mediante el análisis de los patrones de permanencia y cambio definidos, para acercar dicho significado hacia la preservación y cuidado del patrimonio vernáculo construido.} 

\end{enumerate}

% \begin{enumerate}
%
%   \item{Identificar y documentar los sistemas constructivos tradicionales persistentes en las comunidades \emph{El Deca} y \emph{El Buena} ubicadas en el municipio del Cardonal, Hidalgo; mediante levantamientos arquitectónicos con carácter de anteproyecto, geolocalización y generación de una base de datos con la finalidad de contar con un registro que ayude a perpetuar el conocimiento y características de estos sistemas constructivos.}
%
%   \item{Analizar y comparar el significado actual de los sistemas constructivos tradicionales en relación a sus aspectos culturales, cosmogónicos, económicos y de transformación arquitectónica entre personas de origen Hñahñu que se encuentran viviendo en contextos opuestos (rural-ciudad), mediante entrevistas semiestructuradas...}
%
%   \item{Determinar la significación actual de la vivienda tradicional, mediante el análisis de los patrones de permanencia y cambio definidos, para acercar dicho significado hacia la preservación y cuidado del patrimonio vernáculo construido.}
%
% \end{enumerate}
