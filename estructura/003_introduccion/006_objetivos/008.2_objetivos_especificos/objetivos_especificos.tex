\subsection{Objetivos Específicos}

% \begin{enumerate}

%   \item{Documentar histórica y diacronicamente la vivienda tradicional para contar con un registro de los cambios arquitectónicos transcurridos entre los años 1990-2022, mediante levantamientos arquitectónicos, aplicación de entrevistas a profundidad y abiertas a los habitantes, fotografías.} 

%   \item{Analizar las trasformaciones arquitectónicas detectadas más relevantes, producto de la incidencia de la migración, mediante la comparativa de distribución espacial y formas de habitar, así como uso de materiales para la identificación de patrones de cambio y permanencia.}

%   \item{Determinar la significación actual de la vivienda tradicional, mediante el análisis de los patrones de permanencia y cambio definidos, para acercar dicho significado hacia la preservación y cuidado del patrimonio vernáculo construido.} 

% \end{enumerate}

% \begin{enumerate}

%   \item{Identificar y documentar los sistemas constructivos tradicionales persistentes en las comunidades \emph{El Deca} y \emph{El Buena} ubicadas en el municipio del Cardonal, Hidalgo; mediante levantamientos arquitectónicos con carácter de anteproyecto, geolocalización y generación de una base de datos con la finalidad de contar con un registro que ayude a perpetuar el conocimiento y características de estos sistemas constructivos.}

%   \item{Analizar y comparar el significado actual de los sistemas constructivos tradicionales en relación a sus aspectos culturales, cosmogónicos, económicos y de transformación arquitectónica entre personas de origen Hñahñu que se encuentran viviendo en contextos opuestos (rural-ciudad), mediante entrevistas semiestructuradas...}

%   \item{Determinar la significación actual de la vivienda tradicional, mediante el análisis de los patrones de permanencia y cambio definidos, para acercar dicho significado hacia la preservación y cuidado del patrimonio vernáculo construido.}

% \end{enumerate}

\begin{enumerate}
	\item{Identificar qué métodos constructivos de la vivienda vernácula hñähñu aún persisten en dos comunidades rurales del Valle del Mezquital, Hidalgo.}

	\item{Examinar cómo se han modificado de 1990 a 2024 los métodos constructivos tradicionales de la vivienda vernácula hñähñu del Valle del Mezquital.}

	\item{Analizar las modificaciones constructivas y simbólicas de la vivienda vernácula hñähñu del Valle del Mezquital.}

	\item{Evaluar los valores patrimoniales más afectados por la pérdida del saber constructivo hñähñu.}

	\item{Determinar el papel de factores económicos (costos, mano de obra) en la conservación de construcciones tradicionales.}
\end{enumerate}
