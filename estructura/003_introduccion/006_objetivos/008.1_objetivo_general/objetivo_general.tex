\subsection{Objetivo general}
%1. Analizar los sistemas constructivos tradicionales en busca de los factores que causen su deterioro con la finalidad de generar soluciones de preservación, sin alterar ell método constructivo.

%2. Analizar los sistemas constructivos tradicionales en las comunidades de El Deca y El Buena ubicadas en el municipio de El Cardonal en Hidalgo, con el fin de detectar los factores de deterioro y perdida de estos sistemas.

%3. Analizar los sistemas constructivos tradicionales característicos de las comunidades El Deca y El Buena ubicadas en el municipio de El Cardonal estado de Hidalgo, para la identificación de los factores de deterioro y perdida de estos sistemas y de los conocimientos tradicionales de construcción.

%4. Identificar los factores de deterioro y perdida de los conocimientos y sistemas constructivos tradicionales característicos de las comunidades El Deca y El Buena ubicadas en el municipio de El Cardonal estado de Hidalgo, mediante el análisis de los sistemas constructivos 

%El objetivo general tiene que contestar a 3 prentas ,así como no ser muy largo y aglomerar el propocito principar de la investigación.

%\textbf{¿Qué?} = Identificar los factores de deterioro, perdida de conocimientos y sistemas constructivos tradicionales más representativos en las viviendas de las comunidades \emph{El Deca} y \emph{El Buena} en el municipio de \emph{El Cardonal} estado de Hidalgo, 

%\textbf{¿Cómo?} = mediante el análisis de los sistemas, cambios socioculturales y experimentación,

%\textbf{¿Para?} = para llevar a cabo acciones de conservación tanto en temas constructivos y conocimienos generacionales sobre el tema.

%\subsubsection{Objetivo completo

%Identificar los factores de deterioro y perdida de conocimientos sobre los sistemas constructivos tradicionales mas representativos de las comunidades \emph{El Deca} y \emph{El Buena} en el municipio de \emph{El Cardonal} estado de Hidalgo, mediante el análisis de las viviendas sobrevivientes construidos; para llevar a cabo documentación y acciones de prevención ante estos factores.\\

%Identificar los factores de deterioro, perdida de conocimientos y sistemas constructivos tradicionales más representativos en las viviendas de las comunidades \emph{El Deca} y \emph{El Buena} en el municipio de \emph{El Cardonal} estado de Hidalgo, mediante el análisis de los sistemas, cambios socioculturales y experimentación, para llevar a cabo acciones de conservación tanto en temas constructivos y conocimienos generacionales sobre el tema.

% Analizar el fenómeno de transformación arquitectónico y el resignificación actual de la vivienda tradicional, producto de la incidencia de la migración en las comunidades \emph{El Deca} y \emph{El Buena} en el municipio de \emph{El Cardonal}, Hidalgo, entre los años de 1990-2024, para la identificación de patrones de permanencia y cambio mediante un enfoque sincrónico y etnográfico.

% Analizar las modificaciones y permanencias en los sistemas constructivos tradicionales que aún persisten en dos comunidades rurales del Valle del Mezquital, Hidalgo así como las perdidas simbólicas y culturales que esto implica.

% Analizar en dos comunidades rurales del Valle del Mezquital, Hidalgo, las modificaciones y permanencias de los sistemas constructivos vernáculos, y evaluar implicaciones culturales y simbólicas que dichas transformaciones conllevan.

Analizar las modificaciones y permanencias de los sistemas constructivos tradicionales que aún persisten en dos comunidades rurales del Valle del Mezquital, Hidalgo, así como las pérdidas simbólicas y culturales asociadas a estos cambios.
