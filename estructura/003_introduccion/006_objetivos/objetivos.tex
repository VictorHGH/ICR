\subsection{Objetivos}

% \textbf{Objetivo general}
% Analizar la transformación intergeneracional del saber constructivo hñähñu entre 1990 y 2024 en las comunidades El Deca y El Buena, municipio de El Cardonal, Hidalgo, considerando tanto su dimensión técnica como simbólica, y evaluar la pertinencia, el alcance y las limitaciones de los procesos participativos vigentes para revitalizar la arquitectura vernácula como patrimonio vivo en contextos contemporáneos.

% \vspace{1em}
% \textbf{Objetivos específicos}

% \begin{enumerate}
% 	\item \textbf{Documentar, desde una perspectiva técnico-cultural,} las prácticas constructivas tradicionales hñähñu, así como los mecanismos sociales que han sostenido su transmisión (faenas, rituales domésticos), integrando esquemas arquitectónicos, entrevistas, registro audiovisual y croquis afectivos que revelen su valor simbólico y comunitario. 

% 	\item \textbf{Analizar los sistemas constructivos tradicionales hñähñu} vigentes en las comunidades estudiadas, considerando sus características técnicas, materiales y simbólicas, así como su adaptación al entorno natural, su funcionalidad habitacional y su valor como expresión cultural local, con el fin de aportar elementos para su valoración, clasificación y conservación integral. 

% 	\item \textbf{Evaluar las estrategias institucionales y comunitarias actuales} orientadas a la conservación de la vivienda vernácula (programas participativos, talleres, redes de maestros albañiles), identificando sus aportes y limitaciones, y proponiendo criterios para el diseño de modelos de gestión más inclusivos, éticos y técnicamente sostenibles. 
% \end{enumerate}

\textbf{Objetivo general}
Analizar las modificaciones y permanencias de los sistemas constructivos tradicionales que aún persisten en dos comunidades rurales del Valle del Mezquital, Hidalgo, así como las pérdidas simbólicas y culturales asociadas a estos cambios.

\vspace{1em}
\textbf{Objetivos específicos}
\begin{enumerate}
	\item{Identificar qué métodos constructivos de la vivienda vernácula hñähñu aún persisten en dos comunidades rurales del Valle del Mezquital, Hidalgo.}

	\item{Examinar cómo se han modificado de 1990 a 2024 los métodos constructivos tradicionales de la vivienda vernácula hñähñu del Valle del Mezquital.}

	\item{Analizar las modificaciones constructivas y simbólicas de la vivienda vernácula hñähñu del Valle del Mezquital.}

	\item{Evaluar los valores patrimoniales más afectados por la pérdida del saber constructivo hñähñu.}

	\item{Determinar el papel de factores económicos (costos, mano de obra) en la conservación de construcciones tradicionales.}
\end{enumerate}
