\subsection{Objetivos}

\textbf{Objetivo general}

Analizar la erosión intergeneracional del saber constructivo hñähñu (1990-2024) en las comunidades El Deca y El Buena, municipio de El Cardonal, Hidalgo, y evaluar la pertinencia y el alcance de los procesos participativos actualmente vigentes para revitalizar la transmisión de la técnica vernácula y su uso habitacional contemporáneo.

\vspace{1em}
\textbf{Objetivos específicos}

\begin{enumerate}
  \item \textbf{Documentar, de forma diacrónica,} las prácticas constructivas tradicionales y sus mecanismos de transmisión (faenas, maestro-oficial, rituales domésticos), mediante levantamientos arquitectónicos, registro audiovisual de procesos y entrevistas semiestructuradas a propietarios de construcciones tradicionales.

  \item \textbf{Analizar los factores socio-culturales, económicos y normativos} que han propiciado la discontinuidad del saber (escuela formal, mercado de materiales, políticas de vivienda, migración, turismo) valorando su peso relativo a través de matrices causales.

  \item \textbf{Evaluar la eficacia de los lineamientos y programas participativos existentes} (talleres de capacitación, apropiación de sus tradiciones, maestros albañiles) para la conservación de la vivienda vernácula, identificando buenas prácticas, limitaciones operativas y oportunidades de mejora.
\end{enumerate}
