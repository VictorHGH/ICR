\subsection{Estado del conocimiento sobre la vivienda vernácula hñähñu y sus transformaciones constructivas y patrimoniales}

El estudio de la vivienda vernácula mexicana se ha abordado desde la arquitectura, la antropología y los estudios patrimoniales, pues su materialidad condensa saberes constructivos, valores simbólicos y estrategias de adaptación ambiental \cite{rudofsky1964, zarate2009arquitectura}.  Durante décadas se la describió como un tipo de arquitectura marginal al canon disciplinario; hoy se valora su papel en la reproducción cultural y la construcción de identidades locales \cite{gonzalez2017arquitectura, icomos1999carta}.

La institucionalización patrimonial se consolidó con la Convención UNESCO de 1972, que extendió la protección a paisajes culturales \cite{UNESCO1972}, y con la Carta del Patrimonio Vernáculo Construido (1999), que exige participación comunitaria.  Estudios comparativos evidencian, sin embargo, tensiones entre tales directrices y las realidades locales: la adaptación de los criterios de ``autenticidad'' e ``integridad'' al tejido vivo de los asentamientos rurales sigue siendo un desafío \cite{venecia1964card, icomos1999carta, prats1998concepto}.

\subsubsection{Transformaciones materiales, migración y modernidad}

La pérdida acelerada de estas formas de habitar se vincula a la modernización y a la aspiración a modelos urbanos. \cite{alvarez2003maguey}, \cite{juarez2018transformacion} y \cite{boils2010remesas} documentan cómo las remesas financian ``neoviviendas'' de block y concreto; la estética importada se yuxtapone a prácticas productivas todavía rurales.  La salida de jóvenes interrumpe la transmisión de técnicas como mezcla de suelos, uso de maguey, faenas colectivas descrita por Torres y Azevedo (\citeyear{torres2021transmision}).  El resultado es una ruptura generacional que interrumpe la transmisión del saber constructivo y desarticula la relación entre lo material y lo simbólico.

Esta transformación es evidente en comunidades como El Buena y El Deca. Al ingresar a sus centros, predominan edificaciones construidas con materiales industrializados, muchas de ellas inacabadas o en obra negra, que configuran paisajes grises y homogéneos. Estas construcciones reflejan un cambio profundo que no solo afecta a estas localidades, sino que se extiende por todo el Valle del Mezquital.

\subsubsection{Sostenibilidad y vivienda vernácula}

El discurso contemporáneo de la sustentabilidad ha revalorizado la arquitectura tradicional por su bajo impacto ambiental y su pertinencia bioclimática. \cite{gudynas2010desarrollo} distingue tres corrientes de sustentabilidad; la ``fuerte'' y la ``súper-fuerte'' reconocen valores culturales y ecológicos inseparables. \cite{larraga2014sust} sistematizan la eficiencia energética de paredes de tierra cruda, mientras \cite{chang2010patrimonio} subrayan el carácter holístico del patrimonio biocultural.  No obstante, varios autores alertan de una ``ecologización superficial'': la copia formal de técnicas tradicionales sin el saber que las sustenta puede producir resultados ineficientes o inseguros \cite{gandara2000, sanchez2016adobeBTC}.  Existe, por tanto, un vacío entre la retórica verde y la praxis constructiva real.

Este vacío se vuelve especialmente visible en contextos rurales como el Valle del Mezquital, donde las soluciones tradicionales siguen ofreciendo potencial de sostenibilidad real, pero carecen de reconocimiento técnico o institucional.

\subsubsection{Crítica sociológica y folclorización}

Dentro de los estudios críticos, \cite{i2008zombi} caricaturiza el patrimonio como ``zombi'': un muerto viviente reanimado para el mercado global.  Villaseñor y Zolla (\citeyear{villasenor2012patrimonio}) denuncian los procesos de patrimonialización dirigidos por instituciones que espectaculariza prácticas locales sin ceder control a sus portadores. \cite{malavassi2017patrimonio} propone leer el patrimonio como construcción social negociada, mientras Hobsbawm y Ranger (\citeyear{hobsbawm1983inventar}) enfatizan la invención moderna de muchas ``tradiciones''.  Estos enfoques permiten problematizar la musealización de la vivienda hñähñu como decorado turístico.

Estas críticas son pertinentes también en el ámbito de la vivienda hñähñu, donde la preservación formal de las construcciones, desvinculada de su uso y de los saberes técnicos que las sustentan, puede derivar en una patrimonialización superficial.

\subsubsection{Enfoques legales y políticos}

En México, la Ley Federal de Protección del Patrimonio Cultural de los Pueblos y Comunidades Indígenas y Afromexicanas \cite{ley2023patrimonio} y la reciente legislación estatal han reforzado el énfasis en derechos culturales colectivos. \cite{sanchez2013legislacion} muestra, sin embargo, la fragilidad de la aplicación en el ámbito local, mientras la Convención de Diversidad Cultural 2005 reintroduce el eje económico de las expresiones tradicionales \cite{UNESCO2005diversidad}. Estas leyes coexisten con regulaciones de desarrollo urbano, turismo y medio ambiente, generando reglas que a veces se contradicen o generan superposiciones normativas. En ese contexto, la vivienda vernácula queda sujeta a distintas presiones, entre la conservación, el uso cotidiano y el interés del mercado.

\subsubsection{Vacíos detectados}

\begin{enumerate}
	\item Predominan estudios monodisciplinares: o bien se analizan las cualidades constructivas, o bien los discursos patrimoniales, sin integrar variables tecnológicas, socio-simbólicas y económicas.

	\item Escasean metodologías de ``valoración profunda'' que triangulen etnografía, análisis de costo-beneficio y diagnóstico constructivo \citep{delatorre2002values}.

	\item Faltan propuestas de conservación participativa que contemplen simultáneamente la transmisión de saberes y la compatibilidad material; las experiencias piloto documentadas son puntuales y no escalables, lo que refuerza la necesidad de estudios locales que documenten tanto la técnica como su dimensión cultural. (Torres y Azevedo \citeyear{torres2021transmision}).

	\item Las políticas legales se enfocan en la ``protección'' pero poco en las dinámicas vivas de uso y transformación; la tensión entre derecho a la vivienda digna y derecho a la preservación cultural apenas se teoriza \citep{herrejon1994}.
\end{enumerate}

En este contexto, la investigación busca contribuir al conocimiento sobre la vivienda vernácula hñähñu mediante el análisis técnico y simbólico de sus transformaciones constructivas en el Valle del Mezquital.
