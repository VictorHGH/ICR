\subsection{Estado del conocimiento sobre la vivienda vernácula y su transformación patrimonial}

El estudio de la vivienda vernácula en México ha sido abordado desde diversas disciplinas, como la arquitectura, la antropología y los estudios patrimoniales, reconociéndola como una manifestación integral del conocimiento tradicional, adaptada al entorno natural y social de cada comunidad. Esta forma de arquitectura ha sido valorizada no solo por su funcionalidad constructiva, sino por su papel en la reproducción cultural y en la conformación de identidades locales. Sin embargo, frente a procesos de modernización, migración y homogeneización urbana, su permanencia se ve cada vez más amenazada. Esta sección revisa los principales enfoques académicos sobre este fenómeno, así como los vacíos investigativos que la presente investigación busca atender.

Diversos autores han valorado la arquitectura vernácula como una forma de habitar profundamente arraigada al territorio y a la cultura de los pueblos. \citeauthor{rudofsky1976arquitectura} la describió como una ``arquitectura sin pedigrí, arquitectura no formal, arquitectura no clasificada'' \citep[p. 9]{rudofsky1976arquitectura}, subrayando su exclusión histórica del discurso arquitectónico oficial. Por su parte, \citeauthor{gonzalez2017arquitectura} plantea que esta arquitectura no surge de cánones estéticos ni académicos, sino que es modelada directamente por quienes la habitan, a partir de sus necesidades vitales, recursos disponibles y tradiciones heredadas. El \cite{icomos1999carta} la reconoce como una expresión fundamental de la identidad comunitaria, estrechamente ligada al territorio y a la diversidad cultural. Esta visión holística también es retomada por \citeauthor{zarate2009arquitectura}, quien resalta la manera en que la vivienda tradicional condensa en su materialidad elementos religiosos, ambientales, simbólicos y sociales.

A pesar de estos reconocimientos, numerosos estudios advierten sobre la transformación y pérdida progresiva de estas formas de habitar. \citeauthor{alvarez2003maguey} identifica cómo el deseo de emular modelos urbanos, impulsado por procesos de urbanización y difusión cultural, conduce a la sustitución de viviendas tradicionales por construcciones industrializadas, aun cuando estas no se correspondan con las prácticas ni necesidades locales. En ese mismo sentido, \cite{gandara2000} señala que el interés contemporáneo por estas arquitecturas suele centrarse únicamente en su potencial ecológico y sustentable, dejando de lado su valor cultural e histórico. El resultado ha sido una valorización parcial que invisibiliza los saberes comunitarios y las formas tradicionales de habitar.

Las dinámicas de migración también han sido identificadas como un factor clave en la transformación de la vivienda vernácula. \cite{baez2012pueblos} documenta cómo, en regiones como el Valle del Mezquital, la migración juvenil hacia Estados Unidos es una estrategia común para la mejora económica de las familias, lo cual genera una ruptura generacional en la transmisión de saberes constructivos. En muchos casos, el retorno de migrantes implica también la incorporación de nuevos referentes arquitectónicos, materiales y aspiraciones de modernidad, que entran en tensión con las formas tradicionales. Sin embargo, como advierte \citeauthor{gandara2000}, este cambio no solo transforma la vivienda física, sino que altera profundamente las formas de vida, la organización familiar y la relación simbólica con el territorio.

En el ámbito regional, los estudios sobre patrimonio en el Valle del Mezquital han tendido a enfocarse en aspectos materiales o paisajísticos. \cite{rodriguezestudio} identifica y clasifica sitios de interés cultural y natural en la zona, pero no aborda el conocimiento inmaterial vinculado a la vivienda, como los saberes constructivos o las prácticas cotidianas del habitar. Esta omisión es significativa, pues como advierte \cite{guerrero2007arquitectura}, muchos de estos conocimientos se transmiten oralmente o mediante la experiencia directa, lo cual dificulta su documentación y amenaza su continuidad. La ausencia de registros escritos o sistematizaciones técnicas hace que, con el paso del tiempo, estas prácticas puedan distorsionarse o desaparecer, perdiéndose parte del patrimonio inmaterial de las comunidades.
