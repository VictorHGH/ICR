\section{Introducción}

%El siguiente documento desarrolla el planteamiento, justificación y forma para poder llevar acabo una investigación relacionada con la vivienda tradicional como patrimonio cultural y tradicional en el Valle del Mezquital, Hidalgo, con el fin de de llegar a la idónea comunicación de resultados, siguiendo las pautas y reglas enmarcadas por la maestría en Ciencias y Artes para el Diseño de la Universidad Autónoma Metropolitana unidad Xochimilco, tomando en cuenta los aspectos culturales de la vivienda, los materiales y zonas donde se localizan los diferentes sistemas constructivos.

% El siguiente documento desarrolla el planteamiento, justificación y forma para poder llevar a cabo una investigación relacionada con las trasformaciones arquitectónicas mas relevantes, producto de la incidencia de la migración y la resignificación de la vivienda tradicional al igual que su importancia como patrimonio cultural en dos comunidades pertenecientes al Valle del Mezquital, Hidalgo, con el fin de de llegar a la idónea comunicación de resultados, siguiendo las pautas y reglas enmarcadas por la maestría en Ciencias y Artes para el Diseño de la Universidad Autónoma Metropolitana unidad Xochimilco.

% Dicho estudio busca analizar los patrones de cambio en el transcurso del tiempo y por ende la perdida de los sistemas constructivos tradicionales a causa de las transformaciones en las formas de habitar y construir en las dos comunidades a estudiar.

% Las formas, funciones y tecnologías utilizadas en las construcciones han sido parte de la historia de la humanidad, y han sido parte de la evolución de la sociedad, por lo que es importante conocer las formas en que se han ido transformando a lo largo del tiempo, para poder entender la forma en que se han ido adaptando a los cambios que se han presentado en los diferentes entornos de la humanidad.

El Valle del Mezquital, Hidalgo, región habitada inicialmente por la cultura otomí, conquistada por los españoles al rededor de 1550, ha logrado preservar gran parte de su cultura y tradiciones que van desde las danzas, música, celebraciones religiosas y vestimentas forman parte de su conexión con el entorno y su capacidad de adaptación a las dificultades que han atravesado a lo largo del tiempo.
