\section{Justificación}
Dichas transformaciones no solo han afectado las formas de habitar, arquitectónicas y sociales, sino también han afectado al paisaje, cultura y tradiciones, por lo que es importante tener documentado dicho proceso, tanto social, constructivo, y teniendo en cuenta que. Sin embargo esta trasferencia de conocimiento se ha visto afectada debido que hoy en día los jóvenes han ido perdiendo el interés por retomar las tradiciones y costumbres de sus comunidades natales. Es por esto que la documentación, análisis y difusión de las trasformaciones arquitectónicas no solo buscan resguardar los conocimientos, sino entender el por que de estos cambios y la resignificación que la vivienda tradicional ha tenido en los últimos años tomando como referencia los datos estadísticos propuestos tanto por el Instituto Nacional de Estadística y Geografía (INEGI) así como estudios relacionados tanto con la arquitectura vernácula y el patrimonio cultural que esta representa.
