\subsection{Síntesis operacional del marco teórico-metodológico}
\label{subsec:sintesis_operacional}

Las secciones precedentes han definido cuatro ejes conceptuales:
(a) patrimonio vernáculo biocultural (Sec.\,\ref{subsec:vernaculo_biocultural}),
(b) tipologías de valor patrimonial (Sec.\,\ref{subsec:tipologias_valor}) y
(c) modelos de transmisión y ruptura del saber (Sec.\,\ref{subsec:modelos_transmision})

\begin{table}[ht]
	\centering
	\caption{Matriz de correspondencia teoría–variables–métodos}
	\vspace{0.25cm}
	\label{tab:matriz_operacional}
	\begin{tabular}{p{0.22\linewidth} p{0.28\linewidth} p{0.23\linewidth} p{0.20\linewidth}}
		\hline
		\textbf{Constructor}                           &
		\textbf{Variable / Indicador clave}            &
		\textbf{Instrumento principal}                 &
		\textbf{Sección de análisis}                     \\
		\hline
		Valor de uso \citep{riegl1903}                 &
		% de viviendas con fogón tradicional &
		Censo arquitectónico                           &
		Resultados 3.1                                   \\
		\hline
		Valor simbólico \citep{delatorre2002values}    &
		Frecuencia anual de rituales domésticos        &
		Registro etnográfico                           &
		Resultados 3.2                                   \\
		\hline
		Valor histórico                                &
		Estratigrafía de fases constructivas           &
		Levantamiento + fotografía                     &
		Resultados 3.3                                   \\
		\hline
		Canal vertical de transmisión                  &
		Edad promedio de ingreso como albañil          &
		Entrevista semiestructurada                    &
		Resultados 4.1                                   \\
		\hline
		Canal oblicuo de transmisión                   &
		Participación en programas de vivienda estatal &
		Encuesta socio-económica                       &
		Resultados 4.2                                   \\
		\hline
		Índice de Vitalidad Técnica (IVT)              &
		Compuesto de 5 indicadores\footnotemark[1]     &
		Matriz de datos                                &
		Discusión 5.1                                    \\
		\hline
	\end{tabular}
\end{table}

\footnotetext[1]{N.º de maestros activos, edad de ingreso, frecuencia
	de faenas, % de cubiertas de penca, % de rituales documentados;
	ponderados 0.20 cada uno.}

\subsubsection{Lógica de integración}

\begin{enumerate}
	\item Los \emph{valores patrimoniales} permiten jerarquizar los
	      componentes de la vivienda a conservar.
	\item Los \emph{canales de transmisión} explican la dinámica
	      intergeneracional del saber técnico.
	\item Los \emph{indicadores de vitalidad} cuantifican la convergencia
	      entre valores y transmisión.
	\item La \emph{coherencia normativa} revela brechas entre la práctica
	      local y los marcos legales.
\end{enumerate}

\subsubsection{Operatividad en campo}

Cada variable se vincula a una técnica de recolección concreta, generando:

\begin{itemize}
	\item Mapas SIG con capas de valor patrimonial y vitalidad técnica,
	      útiles para priorizar intervenciones.
	\item Matrices de recomendaciones participativas contrastadas con los
	      resultados de la revisión normativa.
\end{itemize}

Esta matriz operacional cristaliza el puente entre el marco conceptual
y la metodología: cada categoría teórica se traduce en indicadores
medibles y en un procedimiento de análisis.  De esta manera, el estudio
podrá evaluar empíricamente la hipótesis central sobre la erosión de la
tradición constructiva hñähñu y fundamentar propuestas de revitalización
patrimonial coherentes con su naturaleza biocultural.
