\subsection{Conclusión del marco teórico-conceptual}

El marco desarrollado en las subsecciones precedentes articula cuatro
planos analíticos:

\begin{enumerate}
	\item \textbf{La condición biocultural del patrimonio vernáculo}
	      (\ref{subsec:vernaculo_biocultural}): la vivienda hñähñu
	      integra ecosistema, técnica y simbolismo; su conservación exige
	      estrategias que operen simultáneamente sobre suelo, maguey y
	      transmisión ritual.
	\item \textbf{La pluralidad de valores patrimoniales}
	      (\ref{subsec:tipologias_valor}): la combinación de
	      \cite{riegl1903} y \cite{delatorre2002values} demuestra que la toma de decisiones
	      depende de la jerarquía, a veces conflictiva, entre valor de
	      uso, valor simbólico y valor histórico.
	\item \textbf{Los mecanismos de transmisión/ruptura del saber}
	      (\ref{subsec:modelos_transmision}): el paso del canal vertical
	      al oblicuo reduce la ``densidad simbólica'' de la técnica
	      (Torres y Azevedo \citeyear{torres2021transmision}) y acelera la
	      adopción de materiales industrializados.
	\item \textbf{La matriz operacional} (\ref{subsec:sintesis_operacional}):
	      traduce los conceptos en variables medibles,
	      preparando el terreno para la metodología del Capítulo 3.
\end{enumerate}

\subsubsection{Límites y proyecciones}

El modelo no pretende agotar la complejidad patrimonial, sino
proporcionar una ``guía-mapa'' que oriente la fase empírica.
Se reconocen dos limitaciones:

\begin{enumerate}
	\item No incorpora, por el momento, indicadores de
	      sostenibilidad financiera; ello se abordará
	      cualitativamente en la discusión.
	\item La matriz de valores se basa en tipologías occidentales; se
	      contrastará con categorías que emerjan de los talleres
	      participativos.
\end{enumerate}
