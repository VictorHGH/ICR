\subsection{Patrimonio vernáculo biocultural}
\label{subsec:vernaculo_biocultural}

La Carta del Patrimonio Vernáculo Construido de ICOMOS (1999) define las
estructuras vernáculas como ``expresiones fundamentales de la identidad
de una comunidad y de la diversidad cultural del mundo''; su protección
depende de la continuidad de uso y del apoyo de la población local
\citep{icomos1999carta}.  Esta declaración ayudo a enfocar parte de la importancia de la
monumentalidad hacia el hábitat cotidiano, legitimando casas de adobe,
graneros y pajares como parte del acervo patrimonial.  No obstante,
proteger la \emph{arquitectura vernácula} implica reconocer que su valor
no se agota en la materialidad: la técnica, los recursos ecológicos y
las prácticas simbólicas forman un sistema en conjunto.

\subsubsection{De lo vernáculo a lo biocultural}

La noción de \emph{patrimonio biocultural} surge en la última década
para subrayar la interdependencia entre diversidad biológica y cultural.
Según \citeauthor{chang2010patrimonio}, no puede preservarse la memoria
colectiva sin salvaguardar los ecosistemas que la sostienen
\citep{chang2010patrimonio}.  En el caso hñähñu, los
magueyales, el suelo arcilloso y el clima semidesértico condicionan la
técnica de la vivienda; a su vez, la práctica constructiva mantiene el
paisaje agavero y la biodiversidad asociada.  La perdida del saber,
sea por migración, escolarización o mercado de materiales, provoca
también la reducción de la agrobiodiversidad regional.

\medskip
\noindent\textbf{Tres niveles de articulación biocultural}
Siguiendo el marco de \cite{gudynas2010desarrollo}, la vivienda
vernácula opera en tres escalas:

\begin{enumerate}
	\item \textbf{Ecológica}: selección de suelos, orientación y masa
	      térmica optimizadas para el ambiente árido; uso de maguey como
	      elemento constructivo y fuente de sombra.

	\item \textbf{Socio-técnica}: organización comunal de faenas,
	      reciprocidad de mano de obra y transmisión oral del saber hacer.

	\item \textbf{Simbólica}: la casa como extensión del cuerpo familiar
	      y espacio de rituales (ofrendas agrarias, fiestas patronales).
\end{enumerate}

El carácter biocultural queda plasmado en la unidad paisaje-técnica-
cosmovisión; la pérdida de uno repercute en los otros dos, fenómeno que
este estudio indagará mediante indicadores de vitalidad y
transmisión (Sección \ref{subsec:tradicion_constructiva}).

\subsubsection{Criterios de autenticidad e integridad}

Las instituciones internacionales (Atenas 1931, Venecia 1964,
Burra 1988) elaboraron criterios de autenticidad e integridad pensados
para monumentos de piedra; su aplicación al patrimonio vernáculo exige
reinterpretaciones.  La Carta CIAV 1999 propone evaluar la autenticidad
en función de la \emph{interacción continua} entre comunidad,
construcción y entorno.  Así, sustituir una cubierta vegetal por lámina
puede alterar la autenticidad si rompe la lógica bioclimática y
simbólica; en cambio, introducir refuerzos de tierra-cal compatibles
podría considerarse intervención legítima.  Esta investigación
empleará la matriz de valores de \cite{delatorre2002values} para
estudiar la autenticidad vivencial frente a la material.

\subsubsection{Implicaciones para la metodología}

Conceptualizar la vivienda hñähñu como patrimonio vernáculo biocultural
supone adoptar:

\begin{itemize}
	\item \textbf{Indicadores mixtos}: se registrarán tanto variables
	      constructivas (espesor de muro, índice de reemplazo de pencas)
	      como bio-socio-culturales (densidad de maguey, frecuencia de
	      faenas, presencia de rituales).
	\item \textbf{Escalas de análisis}: la vivienda (micro),
	      el asentamiento (meso) y el paisaje (macro), integrando SIG y
	      cartografías.
	\item \textbf{Participación comunitaria}: la evaluación de valores
	      patrimoniales incluirá talleres y recorridos diagnosticados con
	      portadores de la técnica, coherente con la Carta CIAV y la Ley
	      2023 de patrimonio indígena \citep{ley2023patrimonio}.
\end{itemize}

\subsubsection{Conclusión de la sección}

El marco biocultural permite trascender la visión objetualista de la
vivienda vernácula y abordar su complejidad como proceso ecológico,
técnico y simbólico.  Reconocer esta naturaleza implica que los
programas de conservación no pueden limitarse a restaurar muros, sino
que deben fomentar la vitalidad de los saberes y del paisaje que los
sustenta. De acuerdo con esto, se construirá, en los apartados
siguientes, la tipología de valores patrimoniales (Sec.\,\ref{subsec:tipologias_valor})
y el modelo de transmisión/ruptura del saber (Sec.\,\ref{subsec:modelos_transmision})
que guiarán la interpretación de los resultados.

