\subsection{Tipologías de valor patrimonial}
\label{subsec:tipologias_valor}

Comprender por qué ciertos elementos de la vivienda hñähñu se conservan,
mientras otros se transforman o desaparecen, requiere identificar los
valores patrimoniales que los distintos actores asignan a la
tradición constructiva.  Dos marcos teóricos resultan especialmente
útiles para esta investigación: la tipología clásica de Alois Riegl
(1903) y la matriz sociocultural–económica desarrollada por De la Torre
y Mason (2002).

\subsubsection{Valores según Riegl}

En \textit{El culto moderno a los monumentos}, \citeauthor{riegl1903} distingue seis valores
(\autoref{tab:riegl_valores}), agrupables en dos ejes:

\begin{itemize}
	\item \textbf{Valores de rememoración} (histórico, de antigüedad,
	      conmemorativo-intencional y no-intencional)
	\item \textbf{Valores de actualidad} (artístico y de uso)
\end{itemize}

Estos últimos, sobre todo el \emph{valor de uso}, resultan decisivos
para la vivienda vernácula: si la casa deja de satisfacer necesidades
domésticas contemporáneas, su legitimidad cultural disminuye y se
favorece la sustitución por tipologías industrializadas.

\begin{table}[ht]
	\centering
	\caption{Valores de Riegl y ejemplos en la vivienda hñähñu}
	\vspace{0.25cm}
	\label{tab:riegl_valores}
	\begin{tabular}{p{0.28\linewidth} p{0.65\linewidth}}
		\toprule
		\textbf{Valor}               & \textbf{Manifestación local}                            \\
		\midrule
		Uso                          & Función productiva (corral, magueyera) y
		confort térmico.                                                                       \\
		Artístico                    & Estética de tierra cruda, cromática ocre del
		paisaje semidesértico.                                                                 \\
		Histórico                    & Secuencias de adobes con huellas de reconstrucciones
		coloniales.                                                                            \\
		Antigüedad                   & Pátina de paredes erosionadas que la comunidad asocia a
		“auténtico” y “antiguo”.                                                               \\
		Conmemorativo–intencional    & Nichos votivos dedicados al santo patrono
		familiar.                                                                              \\
		Conmemorativo–no intencional & Vivienda como símbolo de resistencia
		cultural hñähñu.                                                                       \\
		\bottomrule
	\end{tabular}
\end{table}

\subsubsection{La matriz sociocultural-económica de De la Torre y Mason}

El proyecto Getty sobre \textit{Values and Economics of Cultural
	Heritage} propone agrupar los valores en dos grandes dominios:

\begin{enumerate}
	\item \textbf{Socioculturales}: histórico, simbólico, social,
	      estético, espiritual.
	\item \textbf{Económicos}: de uso (turístico, habitacional) y de
	      no-uso (existencia, legado).\citep{delatorre2002values}
\end{enumerate}

La ventaja de esta matriz radica en que cada categoría puede
asociarse a \emph{indicadores} y \emph{métodos de obtención} (entrevista
etnográfica para el valor simbólico, valoración contingente para el
valor legado, etc.).
La investigación adoptará esta lógica para construir el instrumento de
encuesta, donde cada pregunta se vincula a una
celda de la matriz valor-método.

\subsubsection{Convergencia y tensiones}

La articulación entre ambos marcos teóricos (Riegl vs. Getty) permite:

\begin{itemize}
	\item captar la \textbf{multidimensionalidad} de los valores:
	      un mismo elemento, por ejemplo, la cubierta de penca, puede
	      poseer valor de uso, valor estético y valor de antigüedad.
	\item revelar \textbf{conflictos de valor}: la sustitución de la
	      cubierta vegetal por lámina mejora el valor de uso (impermeable,
	      rápida instalación) pero reduce los valores estético e
	      histórico, como documentan
	      \cite{juarez2018transformacion,alvarez2003maguey}.
	\item vincular valores con \textbf{factores de ruptura}: la caída del
	      valor social o simbólico suele preceder al abandono
	      tecnológico (Torres y Azevedo, \citeyear{torres2021transmision}).
\end{itemize}

\subsubsection{Operacionalización para este estudio}

En la \autoref{tab:operacionalizacion_valores} se resume cómo se
traducirán estos conceptos en variables e indicadores que se medirán en
campo.  Esta tabla alimentará la matriz síntesis del apartado 2.6.

\begin{table}[ht]
	\centering
	\caption{Operacionalización de los valores patrimoniales}
	\vspace{0.25cm}
	\label{tab:operacionalizacion_valores}
	\begin{tabular}{p{0.22\linewidth} p{0.35\linewidth} p{0.33\linewidth}}
		\toprule
		\textbf{Valor}        & \textbf{Indicador de campo}                      & \textbf{Técnica de
		recolección}                                                                                  \\
		\midrule
		Uso                   &                                                                       % de viviendas ocupadas que mantienen corral y
		fogón tradicional     & Censo arquitectónico +
		entrevistas                                                                                   \\
		Estético              & Percepción de ``belleza'' de muros de tierra
		(escala Likert)       & Talleres participativos                                               \\
		Histórico             & Número de fases constructivas
		identificadas in~situ & Levantamientos y
		estratigrafía                                                                                 \\
		Antigüedad            & Preferencia por revoques de cal vs. revoco nuevo & Encuesta a
		propietarios                                                                                  \\
		Simbólico             & Ritos domésticos ligados a la vivienda           & Registro
		etnográfico                                                                                   \\
		Económico – uso       & Gasto familiar en materiales industriales vs.
		tradicionales         & Encuesta socioeconómica                                               \\
		Económico – no-uso    & Disposición a pagar por conservar la casa
		antigua               & Valoración contingente                                                \\
		\bottomrule
	\end{tabular}
\end{table}

\subsubsection{Conclusión de la sección}

El diálogo entre la tipología clásica de Riegl y la matriz
sociocultural-económica de De la Torre y Mason provee un esquema robusto
para interpretar la complejidad valorativa de la vivienda hñähñu.  Este
esquema guiará tanto la construcción de los instrumentos de campo como
el análisis de las tensiones identificadas en los capítulos de
resultados y discusión.

