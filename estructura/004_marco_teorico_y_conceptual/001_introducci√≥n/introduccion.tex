El presente capítulo consolida los fundamentos conceptuales que
sustentan la investigación sobre la erosión intergeneracional del saber
constructivo hñähñu.  Retoma y articula aportes de la historiografía
crítica, la antropología del patrimonio y la teoría de la arquitectura
vernácula para construir una ``caja de herramientas'' analítica que
permita:

\begin{enumerate}
  \item Delimitar qué se entiende por \emph{tradición constructiva} y
        cómo se manifiesta como patrimonio cultural inmaterial;
  \item Situar la vivienda hñähñu dentro del paradigma del \emph{patrimonio
        vernáculo biocultural}, donde convergen valores materiales,
        simbólicos y ecológicos;
  \item Precisar las tipologías de valor patrimonial y los modelos de
        transmisión–ruptura del saber que orientarán la lectura de los
        datos empíricos;
  \item Interpretar las tensiones entre la práctica viva y la normativa
        patrimonial vigente, identificando categorías analíticas que
        vinculen lo local con lo global.
\end{enumerate}

Para cumplir estos propósitos, el capítulo se organiza en seis
apartados: la Sección~\ref{subsec:tradicion_constructiva} define la
tradición constructiva como patrimonio inmaterial; la
Sección~\ref{subsec:vernaculo_biocultural} profundiza en el concepto de
patrimonio vernáculo biocultural; la
Sección~\ref{subsec:tipologias_valor} revisa las principales tipologías
de valor; la Sección~\ref{subsec:modelos_transmision} presenta los
modelos de transmisión y ruptura del saber; la
Sección~\ref{subsec:normativa_tensiones} analiza el marco normativo; y
finalmente la Sección~\ref{subsec:sintesis_operativa} sintetiza los
conceptos en una matriz operativa que vincula variables e indicadores
con las preguntas de investigación.

Con esta base teórica se busca garantizar la coherencia interna entre
hipótesis, metodología y análisis de resultados, evitando reduccionismos
que limiten la comprensión del fenómeno estudiado.
