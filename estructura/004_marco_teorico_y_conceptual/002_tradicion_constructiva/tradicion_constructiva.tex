\subsection{Tradición constructiva como patrimonio inmaterial}
\label{subsec:tradicion_constructiva}

La noción de \emph{tradición} ha sido ampliamente debatida en la
historiografía y la antropología.  En su célebre compilación sobre la
\enquote{invención de la tradición}, Hobsbawm y Ranger (\citeyear{hobsbawm1983inventar})
argumenta que muchas prácticas consideradas “antiguas” son, en realidad,
creaciones modernas que otorgan legitimidad a proyectos de identidad
colectiva.  Más que descalificar su autenticidad, esta perspectiva
destaca el carácter \emph{dinámico} y \emph{selectivo} de la
transmisión cultural: las comunidades reelaboran sus saberes para hacer
frente a nuevas condiciones sociales, políticas y económicas.

Siguiendo esa línea, \citeauthor{herrejon1994} define la tradición como
un \emph{conjunto de pautas de acción y significado}
socialmente sancionadas, cuya validez depende de la aceptación
colectiva.\citep{herrejon1994}  Se trata de prácticas que se
mantienen vivas en tanto cumplan una función reconocida dentro del
grupo.  En el caso hñähñu, la arquitectura de tierra y maguey no es
solo un testimonio material, sino un saber técnico-social que articula
identidad, economía doméstica y cosmovisión del territorio
semidesértico.

\subsubsection{De la materialidad al patrimonio inmaterial}

La Convención de la UNESCO sobre Patrimonio Cultural Inmaterial (2003)
subraya que los conocimientos y técnicas
tradicionales, al igual que los rituales o la música, constituyen
``prácticas, representaciones y saberes transmitidos de
generación en generación''.  Aunque la vivienda vernácula se manifiesta
materialmente en muros, cubiertas y patios, lo que reviste valor
patrimonial es el \emph{saber hacer} que permite reproducirla: la
selección del suelo arcilloso, la dosificación de la paja y la cal, la
faena comunal para la recolección o cortar pencas de maguey. Ese saber
se rige por códigos simbólicos (fechas propicias
para la recolección) que otorgan dimensión identitaria al proceso.

Esta perspectiva concuerda con el enfoque biocultural propuesto por
\cite{chang2010patrimonio}: la vivienda
vernácula integra recursos naturales, conocimientos empíricos y
expresiones simbólicas en una sola unidad patrimonial.  Así, la
\emph{tradición constructiva} se entiende como un sistema
socio-ecológico cuya pérdida afecta tanto a la diversidad cultural como
a la biodiversidad local, por ejemplo, la disminución del cultivo de
maguey por falta de demanda constructiva.

\subsubsection{Mecanismos de transmisión}

En el Valle del Mezquital, la continuidad de la técnica se ha sustentado
históricamente en cuatro mecanismos principales:

\begin{enumerate}
	\item \textbf{Faenas comunales} - jornadas colectivas que
	      articulan trabajo, reciprocidad y festividad, donde se pueden llegar a
	      compartir conocimientos sobre sistemas constructivos tradicionales.
	\item \textbf{Relación maestro-oficial} - aprendizaje informal bajo la
	      tutela de un albañil reconocido; el
	      \emph{oficial} imita, practica y, con el tiempo, innova sutilezas
	      técnicas. (Torres y Azevedo, \citeyear{torres2021transmision})
	\item \textbf{Rituales domésticos} - nichos y ofrendas en el interior
	      de las construcciones tradicionales.
	\item \textbf{Oralidad técnica} - refranes, consejos y
	      expresiones en lengua hñähñu que codifican proporciones, ciclos
	      climáticos y tiempos de secado.
\end{enumerate}

La investigación parte de la hipótesis de que estos mecanismos se
encuentran hoy desarticulados; ello explicaría por qué la mera
presencia de materiales tradicionales o incluso la existencia de normas
de conservación no garantizan la preservación de la técnica.

\subsubsection{Criterios de análisis}

Para evaluar la continuidad o pérdida de la tradición constructiva
se adoptarán dos criterios derivados de \cite{UNESCO2005diversidad}:

\begin{itemize}
	\item \textbf{Vitalidad}: frecuencia con la que se realizan las
	      prácticas (número de faenas anuales, proyectos supervisados por
	      maestros tradicionales, etc.).
	\item \textbf{Transmisión}: grado en que los jóvenes adquieren el
	      saber (porcentaje de aprendices menores de 30 años involucrados
	      en obras vernáculas).
\end{itemize}

Ambos indicadores se integrarán posteriormente en las matrices causales
para correlacionarlos con variables socio-económicas y normativas.

\subsubsection{Conclusión de la sección}

Esta revisión muestra que la tradición constructiva es,
ante todo, un proceso cultural vivo cuya continuidad depende de redes de
aprendizaje, de legitimación simbólica y de condiciones económicas
mínimas.  Al adoptar el enfoque de patrimonio inmaterial, la presente
investigación desplaza la atención del \emph{objeto} (la casa de tierra)
al \emph{sujeto colectivo} que lo hace posible, fundamento que
orientará tanto la selección de métodos etnográficos como la lectura de
los datos empíricos.
