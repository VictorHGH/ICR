%--------------------------------------------------------------------
\subsection{Modelos de transmisión y ruptura del saber}
\label{subsec:modelos_transmision}
%--------------------------------------------------------------------

El saber constructivo hñähñu se reproducía, hasta finales del siglo XX,
mediante la participación en faenas comunales, el aprendizaje
maestro-oficial y los rituales domésticos asociados a la
autoconstrucción.  Diversos estudios documentan que, cuando estos
mecanismos se debilitan, la técnica pierde su dimensión simbólica y se
vuelve prescindible frente a opciones industrializadas
\citep{torres2021transmision,gandara2000}.  Para interpretar esta
erosión se distinguen tres \emph{canales de transmisión cultural}:

\begin{itemize}
	\item \textbf{Vertical} – intrafamiliar, predominante hasta la década
	      de 1980.
	\item \textbf{Horizontal} – entre pares, impulsado por programas de
	      autoconstrucción y la difusión mediática de materiales
	      prefabricados.
	\item \textbf{Oblicuo} – desde instituciones (Iglesia, agencias de
	      vivienda, industrias de la construcción) hacia la comunidad.
\end{itemize}

La hipótesis de trabajo sostiene que el cambio progresivo del
canal vertical al oblicuo se relaciona con la reducción de la ``densidad
simbólica'' de la vivienda: se conserva la \emph{receta} material pero se
pierde la lógica ritual y ecológica que la legitimaba.

\subsubsection*{Factores de ruptura}

\begin{enumerate}
	\item \textbf{Económicos}: abaratamiento del block y acceso a crédito
	      para autoconstrucción.
	\item \textbf{Culturales}: prestigio de la estética urbana y
	      estigmatización de la tierra cruda.
	\item \textbf{Institucionales}: exigencias de “vivienda digna” que
	      favorecen acabados industrializados \citep{herrejon2006patrimonio}.
	\item \textbf{Ambientales}: reducción de magueyales y sobre-explotación
	      de suelos aptos para adobe.
	\item \textbf{Educativos}: currículo escolar que prioriza ingeniería
	      civil sobre oficios vernáculos.
\end{enumerate}

Estos factores serán representados mediante un \emph{diagrama causal de
	dominio}.  A cada nodo se asignará un peso derivado de la triangulación
entre encuesta, entrevista y observación participante; el modelo
servirá como base para discutir escenarios de revitalización en el
Capítulo 5.

\subsubsection*{Indicadores de vitalidad y transmisión}

Inspirados en la guía UNESCO sobre patrimonio inmaterial (2008) y la
metodología de \cite{gandara2000}, se utilizarán los siguientes cinco
indicadores:

\begin{enumerate}
	\item Número de maestros albañiles activos por década.
	\item Edad promedio de ingreso como oficial.
	\item Frecuencia anual de faenas comunales.
	\item Porcentaje de viviendas que conservan cubierta de penca.
	\item Proporción de rituales constructivos documentados.
\end{enumerate}

La combinación de indicadores materiales e inmateriales permitirá
estimar el \emph{índice de vitalidad técnica} (IVT) descrito en
\autoref{subsec:tradicion_constructiva}.

\subsubsection*{Conclusión de la sección}

La erosión del saber constructivo hñähñu no obedece a un único detonante
sino a la convergencia de rupturas en los canales de transmisión, los
incentivos económicos y los marcos normativos.  Este marco teórico
proporciona la base operativa para evaluar —en los capítulos de
resultados— la efectividad de los procesos participativos existentes y
proponer estrategias de revitalización coherentes con la dimensión
biocultural del patrimonio vernáculo.

