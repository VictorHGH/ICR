\section{Marco teórico}

\subimport{./001_introducción}{introduccion.tex}
\subimport{./002_tradicion_constructiva}{tradicion_constructiva.tex}
\subimport{./003_patrimonio_vernaculo_biocultural}{patrimonio_vernaculo_biocultural.tex}
\subimport{./004_tipologias_de_valor}{tipologias_de_valor.tex}
\subimport{./005_modelos_transmision}{modelos_transmision.tex}
\subimport{./006_sintesis_operacional}{sintesis_operacional.tex}
\subimport{./007_conclusion_marco}{conclusion_marco.tex}

% \subimport{./contexto_historico}{contexto_historico.tex}
%\subimport{./contexto_social}{contexto_social.tex}
%\subimport{./contexto_artistico}{contexto_artistico.tex}

% \subimport{./009.2_vivienda_vernacula}{vivienda_vernacula.tex}
% \subimport{./009.1_patrimonio_cultural}{patrimonio_cultural.tex}
% \subimport{./7.3_Migracion}{migracion.tex}

%   El patrimonio cultural, ``entendido como todo aquello que socialmente se considera digno de conservación independientemente de su interés utilitario'' \citep[p. 63]{prats1998concepto}, a sido estudiado de diversas maneras, desde el nivel profundo de la definición de cada uno de sus conceptos hasta el significado de importancia global que se le acuña hoy en día, sin embargo, la definición de lo que representa patrimonio puede llegar a ser subjetiva para cada institución, persona o sociedad.
% %
% Los alcances del patrimonio van desde los ámbitos cultural y natural, siendo dirigido a elementos tanto materiales como inmateriales definiendo estos aspectos como ``Todo aquello que caracteriza al hombre como especie autoconsciente que genera mediante dichas manifestaciones, cultura'' \citep{dominguez2004pautas}. Por ende, el ser humano es el mismo que define y delimita los alcances que el patrimonio puede llegar a tener.
% %
% La definición del concepto patrimonio forma parte de una construcción social, la cual se puede dividir en natural, cultural, biocultural, etc. Este tipo de valor, ha llegado a ser búsqueda de instituciones privadas, estatales y gubernamentales que buscan el beneficio económico, social y cultural de los grupos a los que pertenece, sin embargo, esto ha provocado que dicho concepto, muchas veces no pase a ser más que algo monetario o beneficio de carácter político, siendo prueba de esto lo que \cite{i2008zombi} llama el zombie de la modernidad.
% %
% La Organización de las Naciones Unidas para la Educación, la Ciencia y la Cultura, describe que ``...el patrimonio cultural y el patrimonio natural están cada vez más amenazados de destrucción, no sólo por las causas tradicionales de deterioro sino también por la evolución de la vida social y económica que las agrava con fenómenos de alteración o de destrucción aún más terribles''\citep[p. 1]{UNESCO1972}.
% %
% En \citeyear{venecia1964card} como parte del II congreso Internacional de Arquitectos y Técnicos de Monumentos Históricos surge la carta de Venecia donde se plantean estrategias para la restauración y conservación de monumentos al igual que la creación del Consejo Internacional de los Monumentos y los Sitios (ICOMOS). Dicha institución es la encargada de promover el cuidado y asignación de la denominación ``patrimonio cultural en México''.
% %
% Algunos de los acuerdos que se han realizado a nivel mundial son la -Convención para la salvaguardia del patrimonio cultural inmaterial- y la -Convención sobre la protección del patrimonio mundial, cultural y natural de 1972- realizadas por la Organización de las Naciones Unidas para la Educación, la Ciencia y la Cultura \citep{UNESCO2003, UNESCO1972}. Estos han sido puntos claves y de inflexión en el tema del patrimonio cultural e inmaterial, dando pautas, estudiando los problemas, generando políticas y la unión de los países participantes en la búsqueda de la protección del dichos patrimonios.
% %
% En 1972 fue creado el ``Comité del Patrimonio Mundial intergubernamental'' conformado inicialmente por quince países. Este comité se reúne cada año con el motivo de de servir como órgano consultivo en los procesos de retiro y agregación a la lista del patrimonio mundial, monitoreo, presentación de informes entre otros puntos.
% %
% % Estos procesos han logrado colocar a los monumentos históricos como patrimonio edificado, sin embargo, la moumentalidad es lo que se suele entender globalmente como patrimonio, el acercamiento a los sistemas constructivos que conformaban las bases de cada sociedad e identidad viene desde el hogar, vivienda o casa, donde las personas se desarrollan de manera íntima y familiar. 
% %
% Teniendo en cuenta lo anterior, es necesario describir que la necesidad de sociedad influye en todas las facetas del ser humano ya que ``los aspectos culturales y naturales que revisten a todas las civilizaciones son fundamentales para entender sus prácticas cotidianas y festivas, presentes y pasadas. Estos aspectos se emplazan en el territorio, dibujando un entramado civilizatorio que distingue a cada tiempo y a cada sociedad de los demás''\cite*[p. 1]{rodriguezestudio}
% %
% Este tipo de patrimonio es algo a lo que no se le ha dado la importancia que se merece, especialmente a la documentación y estudio de la cuestión vernácula de la vivienda, el cual, debido a su importancia como lo menciona el \cite{icomos1999carta}, es una ``expresión fundamental de la identidad de una comunidad, de sus relaciones con el territorio y al mismo tiempo, la expresión de la diversidad cultural del mundo''.
% %
% % Si bien los sistemas constructivos tradicionales, arquitectura vernácula, vivienda tradicional y vivienda popular son algunos de los términos con los que se conocen los sistemas que los pueblos originarios utilizaban para la construcción de sus viviendas. La mas utilizadas fue descrita por Bernard \cite*{rudofsky1976arquitectura} el cual acuño el termino vernáculo a este tipo de construcciones y fue uno de los principales iniciantes en la apertura académica en el campo de este tipo de arquitectura.
% %
% % Este tipo de arquitectura ``no responde a estilos, no representa épocas, no necesita de arquitectos, son quienes las habitan los encargados de modelarlas, ha estado allí, testigo de la cultura de los hombres`` \citep[p. 1]{gonzalez2017arquitectura}. Siendo parte del patrimonio de las comunidades y asentamientos humanos desde sus inicios, inventores de este tipo de arquitectura que no surge en busca de estética, proporción o cualquier otro objetivo, si no por la necesidad de un refugio el cual era construido con los materiales que cada localidad y comunidad tenia a su alcance, haciendo lo mejor que se podía conseguir con lo que proporcionaba la naturaleza.
% %
% \citeauthor{rapoport1972vivienda} hace una de las primeras descripciones sobre la importancia de la vivienda en la cultura de la sociedad, desde la descripción de la tecnología y sus implicaciones en los cambios formales de las construcciones, hasta los usos que cada grupo es capaz de darle a los materiales y sistemas, creando conexiones profundas con la naturaleza o generando estatus relacionados con dichos conceptos.
% %
% Valeria prieto \citeyear{carrillo1978vivienda} nos describe como la ``vivienda campesina'' como la nombra ella, tiene una unión muy profunda con la naturaleza y el ser humano, ya que es en ella donde vivimos tristezas, felicidad, muerte, vida, etc. Pero tenemos que tener en cuenta que para nuestros días, este tipo de vivienda ha sido influenciada por la historia, refiriéndonos a todos los procesos sociales y económicos por los pasa cualquier sociedad a lo largo del tiempo.
% %
% \citeauthor{lopez1993arquitectura} nos describe algunas formas de acercamiento al conocimiento e historia de la arquitectura, desde los cronistas, murales y pinturas de la antigüedad debido a que se han encontrado vestigios en ruinas, documentos pictográficos, etc. De igual manera remarca la importancia de los materiales locales, forma, tecnología y función de la que fueron capaces nuestros antepasados.
% %
% Conociendo la importancia natural e histórica de los sistemas constructivos tradicionales también podemos encontrar a \citeauthor{zarate2009arquitectura} quien recalca la relación holística de la vivienda, aglomerando las creencias, cultura, naturaleza, religión y tiempo.
% %
% Otro tema por tratar relacionado con los sistemas constructivos tradicionales es el valor inmaterial ya que este forma parte importante del patrimonio cultural pero el origen de estos ``es difícil rastrear ya que en parte es el resultado de siglos durante los cuales se daba por hecho que las sociedades heredan objetos del pasado que es necesario conservar, sin que esto obligara a una reflexión profunda de los valores y razones para su conservación'' \citep[p. 6]{villasenor2011valor}.
%
% Unido a lo anterior, el valor inmaterial, como todo valor postulado hacia cualquier cosa, entidad, o pensamiento, es algo propuesto y generado por la sociedad, visto muchas veces fuera de los ojos de las personas que habitan o desarrollan el valor de las cosas.
%
% Los elementos arquitectónicos y sistemas constructivos tradicionales en comunidades rurales, al formar parte de su historia, registro viviente de los cambios sociales, culturales, intimidad de las personas, formándose alrededor de diferentes creencias, costumbres y tradiciones se ha transformado en algo característico del ser humano por lo que forma parte de su patrimonio cultural, pero debido a su decadencia y desuso, está pasando a formar solo parte de la historia.
%
% Dichas transformaciones no solo han afectado las formas de habitar, arquitectónicas y sociales, sino también han afectado al paisaje, cultura y tradiciones, por lo que es importante tener documentado dicho proceso, tanto social, constructivo, y teniendo en cuenta que: 
%
% \begin{center}
%     \begin{minipage}{0.9\linewidth}
%         \vspace{4pt}%margen superior de minipage
%         {\small...los conocimientos tradicionales presentan el inconveniente de que, por haber sido transferidos oralmente y mediante experiencias vivénciales de una generación a otra, rara vez se cuenta con documentos que permitan su caracterización y difusión. Además, como sucede con otras costumbres populares, es común que con el paso del tiempo vayan recibiendo influencias externas o alteraciones que en ocasiones acaban por desvirtuar sus bases originales.
%         }
%         \begin{flushright}
%             \citep[p. 182]{guerrero2007arquitectura}
%         \end{flushright}
%         \vspace{4pt}%margen inferior de la minipage
%     \end{minipage}
% \end{center}
%
% Junto con pasar de ser sensaciones que forman parte del patrimonio inmaterial, que se desarrollan hasta lo práctico, a lecturas donde no se puede transmitir lo mismo que la vivencia y experiencia misma del habitar en un hogar tradicional, en lugar de una búsqueda que implique retomar sus usos como una alternativa sustentable y competente.
%
% ``La sociedad centromexicana durante el Posclásico Tardío tenía una estructura jerárquica. La familia era el núcleo básico. Los conjuntos de familias formaban una agrupación que los españoles llamaban la cuadrilla. Varias cuadrillas formaban un barrio. Los barrios se unían dentro de los señoríos. Algunos señoríos formaban confederaciones estratégicas.'' \citep[p. 156]{Otomies} hoy en día, la estructura familiar continua siendo hasta cierto punto, similar a la que se describe, ya que todavía se suele contar con un sistema patriarcal, en el que el hombre de la casa provee el sustento a toda la familia, y enseña a los hijos varones el oficio o da la oportunidad a que se desenvuelvan en el entorno laboral, mientras las mujeres se dedican al cuidado de la casa y a la enseñanza a las niñas del hogar.
% %
% ``El Valle del Mezquital, el cual conforma una macroregión, compuesta por 27 municipios, que se caracteriza por un clima semidesértico, muy caliente durante el día y con bajas temperaturas por la noche. Hay una escasa precipitación y la vegetación es principalmente xerófila. La temperatura promedio de 13 grados centígrados y de 21 grados en los meses mas calurosos. La precipitación anual promedio es de 409 milímetros. Se clasifica la región en tres subregiones, con características de suelo diferentes, lo que hace que su población se relacione con el entorno de distinta manera`` \citep[p. 56]{alvarez2003maguey}.
% %
% % Esto nos da las primeras directrices en busca de una catalogación de los diferentes sistemas constructivos tradicionales y el por que de la variedad de los mismos en la región y sus relaciones con el clima, temperatura y materiales locales ya que desde el inicio, los sistemas tradicionales se caracterizaron por ser mas un refugio que lo que hoy conocemos como vivienda, pero se fueron trasformando a travez del tiempo hasta llegar a aglomerar los usos y costumbres que hoy los rodean.
% %
% ``En los contextos arqueológicos, en muestras de adobes y tierra, se identifican especies como ahuehuete (Taxodium mucronatum), pino (Pinus sp.), sauce (Salix), ciprés (Cupressus), encino (Quercus), mezquite (Prosopis), maguey (Agave sp), nopal (Opuntia), huizache (Acacia), cardón (Ilex o Lemaireocereus), biznagas (Echinofossulocactus), yuca o palma (Yucca)... Muestra de la variedad de materiales con los que se solía construir en la región, siendo prueba de la capacidad de las poblaciones que habitaron la zona y su capacidad de adaptación a los climas secos y áridos del lugar.``\citep[p. 5]{aguilar2009}
% %
% Actualmente el patrimonio cultural de la región se caracteriza por su naturaleza y lugares de interés y conocidos por los lugareños, como se puede leer en el trabajo ``Estudio e identificación del patrimonio cultural y natural en el valle del mezquital''\citep{rodriguezestudio} el principal objeto en cualquier relación social-patrimonial son las personas las cuales mantienen estos lugares, conocimientos y tradiciones, ya que sin las personas que los visiten y valoren, estos se abandonarían y eventualmente se perderían.
% %
% Hace falta recalcar que en su estudio, no se encuentra el tipo de conocimiento inmaterial, ya que las personas son consientes del patrimonio cultural material refiriendo a lugares específicos, pero en ningún momento se habla del patrimonio inmaterial implícito en sus tradiciones, fiestas y en este caso, los sistemas constructivos tradicionales, cuyo conocimiento pasa de manera generacional a los más jóvenes.
% %
% % ``La historia prehispánica del grupo Otomí es muy oscura, creyéndose durante algún tiempo que tal grupo étnico es uno de loa más antiguos de Mesoaméricica`` \cite[p. 67]{GuerreroGuerrero1983}. Desde Cuicuilco y Copilco, hasta el Valle del Mezquital, lo Otomies o mejor conocidos por ellos mismos como Hñähñus, son una de las culturas con mas enigma en su origen y desglose por desde el sur del Valle de México hasta el Valle del Mezquital pero no siendo estos los primeros habitantes del lugar.
% %
% % El estudio a profundidad del origen de este grupo étnico se dificulta debido a la larga lista de grupos sociales, cambios que a sufrido la región y la perdida o falta de registros de los mismos a lo largo de la historia.
% %
% % ``A pesar de que el grupo otomí se asentó en la región árida del Valle del Mezquital, éste, por su ubicación y condiciones ecológicas, encerraba una variedad de hábitats y nichos. Su riqueza se hacía manifiesta en la diversidad de cactáceas, bosques de pino piñonero, agaves, yucas, mezquites, además de la nutrida fauna que frecuentaba sus chaparrales. Todo ello, combinado con el conocimiento ancestral adquirido por su contacto con los grupos nahuas, les abría la posibilidad de utilizarlo de manera variada, intensiva y acorde con los ciclos naturales.`` \citep{granados2004agricultura}
% %
% % El Valle ha sido una región en constante movimiento desde sus inicios, desde los primeros asentamientos hasta lo que lo conforma hoy en día, pero siempre manteniendo los magníficos paisajes y su vegetación, siendo esto algo por lo que las personas de muchos lugares visitan los poblados de la región.
% %
% % ``La forma de habitar hoy en día se basa en el predio para cada familia, separada de las demás viviendas, a pesar de que las personas suelen conocerse por vínculos familiares o de amistad. Dejando lugar si el espacio lo permite, siempre espacio para el lugar de siembra''\cite*{GuerreroGuerrero1983}, esta forma de habitar se ve afectada en los principales municipios, donde la tipología arquitectónica a pasado a parecerse aún mas a la vista en las ciudades, en donde en conjunto con la tradición en el patrilocalismo donde el patriarca de la familia hereda a los hijos parte del terreno donde habitaron y las mujeres e hijo se mudan a dichos terrenos haciendo que los espacios cada vez se vean reducidos, provocando los cambios topológicos.
