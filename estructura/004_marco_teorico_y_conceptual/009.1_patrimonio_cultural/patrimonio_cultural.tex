\subsection{Patrimonio cultural}
% El patrimonio cultual, ``entendido como todo aquello que socialmente se considera digno de conservación independientemente de su interés utilitario'' \citep[p. 63]{prats1998concepto}, a sido estudiado de diversas maneras, desde el nivel profundo de la definición de cada uno de sus conceptos por separado hasta el significado de importancia global al cual se le acuña hoy en día, los alcances del patrimonio van desde los ámbitos cultural y natural, siendo dirigido a elementos tanto materiales como inmateriales. Los elementos arquitectónicos y sistemas constructivos tradicionales en comunidades rurales, al formar parte de su historia, registro viviente de los cambios sociales, culturales, testigo de la intimidad de las personas, formándose alrededor de diferentes creencias, costumbres y tradiciones se ha transformado en algo característico del ser humano por lo que forma parte de su patrimonio cultural, pero debido a su decadencia y desuso, está pasando a formar solo parte de la musealización y documentación, junto con pasar de ser sensaciones que forman parte del patrimonio inmaterial, que se desarrollan hasta lo práctico, a lecturas donde no se puede transmitir lo mismo que la vivencia y experiencia misma del habitar en un hogar tradicional, en lugar de una búsqueda que implique retomar sus usos como una alternativa sustentable y competente.

% El valor intrínseco cuyo origen ``es difícil rastrear ya que en parte es el resultado de siglos durante los cuales se daba por hecho que las sociedades heredan objetos del pasado que es necesario conservar, sin que esto obligara a una reflexión profunda de los valores y razones para su conservación'' \citep{villasenor2011valor}. Pero como todo valor postulado hacia cualquier cosa, entidad, o pensamiento, es algo propuesto y generado por la sociedad, visto muchas veces desde el exterior incluso fuera de los ojos de las personas que habitan o desarrollan el valor intrínseco de las cosas. Siendo esto parte de la construcción social del patrimonio. El valor patrimonial, ha llegado a ser búsqueda de instituciones privadas, estatales y gubernamentales que buscan beneficiarse de el, haciendo que el término patrimonio, no pase a ser mas que algo monetario.
%
% La -Convención para la salvaguardia del patrimonio cultural inmaterial en 2003-\citep{UNESCO2003} y la -Convención sobre la protección del patrimonio mundial, cultural y natural de 1972- realizadas por la Organización de las Naciones Unidas para la Educación, la Ciencia y la Cultura, han sido puntos claves y de inflexión en el tema del patrimonio cultural e inmaterial, dando pautas, estudiando los problemas, generando políticas y la unión de los países participantes en la búsqueda de la protección del patrimonio Cultural, en 1972 creando el ``Comité del Patrimonio Mundial intergubernamental'' conformado primeramente por quince países. Este comité se reúne cada año con el motivo de de servir como órgano consultivo.
%
% ``Los aspectos culturales y naturales que revistes a todas las civilizaciones son fundamentales para entender sus prácticas cotidianas y festivas, presentes y pasadas. Estos aspectos se emplazan en el territorio, dibujando un entramado civilizatorio que distingue a cada tiempo y a cada sociedad de los demás'' \citep{rodriguezestudio}. Estos aspectos se han visto afectados por el crecimiento y cambio rápido de las sociedades por la globalización, avances tecnológicos y el crecimiento acelerado de la población, especialmente en los países en desarrollo.
%
% Hoy en día, la documentación del patrimonio cultural es algo a lo que no se le ha dado la importancia que se merece, especialmente a la documentación del patrimonio vernáculo, el cual, debido a la importancia que tiene para toda comunidad. La idea de su documentación y catalogación viene fundamentada a que cada localidad, como se mencionó anteriormente, tiene factores diferentes en los ámbitos: ambientales, sociales y culturales, ya que cada grupo social fue generando y transformando dichos sistemas vernáculos a su forma de habitar, necesidades, territorio y materiales a su alcance llegando ser una muestra de la capacidad creativa de la raza humana, mostrar los valores humanos y culturales de una determinada época de la historia, ser testigo de la cultura de las comunidades, ejemplificando un tipo de arquitectura que a pesar de ser diversa en sus métodos constructivos, tienen la misma base fundamental, basada en tierra y con materiales locales, ejemplo del hábitat de humanos con culturas pasadas y está asociada a tradiciones, costumbres y conocimientos vivos desde la antigüedad.

El patrimonio cultural, ``entendido como todo aquello que socialmente se considera digno de conservación independientemente de su interés utilitario'' \citep[p. 63]{prats1998concepto}, a sido estudiado de diversas maneras, desde el nivel profundo de la definición de cada uno de sus conceptos hasta el significado de importancia global que se le acuña hoy en día, sin embargo, la definición de lo que representa patrimonio puede llegar a ser subjetiva para cada institución, persona o sociedad.

Los alcances del patrimonio van desde los ámbitos cultural y natural, siendo dirigido a elementos tanto materiales como inmateriales definiendo estos aspectos como ``Todo aquello que caracteriza al hombre como especie autoconsciente que genera mediante dichas manifestaciones, cultura'' \citep{dominguez2004pautas}. Por ende, el ser humano es el mismo que define y delimita los alcances que el patrimonio puede llegar a tener.

La definición del concepto patrimonio forma parte de una construcción social, la cual se puede dividir en natural, cultural y biocultural. Este tipo de valor, ha llegado a ser búsqueda de instituciones privadas, estatales y gubernamentales que buscan el beneficio económico, social y cultural de los grupos a los que pertenece, sin embargo, esto ha provocado que dicho concepto, muchas veces no pase a ser más que algo monetario o beneficio de carácter político, siendo prueba de esto lo que \cite{i2008zombi} llama el zombie de la modernidad.

La Organización de las Naciones Unidas para la Educación, la Ciencia y la Cultura, describe que ``...el patrimonio cultural y el patrimonio natural están cada vez más amenazados de destrucción, no sólo por las causas tradicionales de deterioro sino también por la evolución de la vida social y económica que las agrava con fenómenos de alteración o de destrucción aún más terribles''\citep[p. 1]{UNESCO1972}.

En \citeyear{venecia1964card} como parte del II congreso Internacional de Arquitectos y Técnicos de Monumentos Históricos surge la carta de Venecia donde se plantean estrategias para la restauración y conservación de monumentos al igual que la creación del Consejo Internacional de los Monumentos y los Sitios (ICOMOS).

Algunos de los acuerdos que se han realizado a nivel mundial son la -Convención para la salvaguardia del patrimonio cultural inmaterial- y la -Convención sobre la protección del patrimonio mundial, cultural y natural de 1972- realizadas por la Organización de las Naciones Unidas para la Educación, la Ciencia y la Cultura \citep{UNESCO2003, UNESCO1972}. Estos han sido puntos claves y de inflexión en el tema del patrimonio cultural e inmaterial, dando pautas, estudiando los problemas, generando políticas y la unión de los países participantes en la búsqueda de la protección del dichos patrimonios.

En 1972 fue creado el ``Comité del Patrimonio Mundial intergubernamental'' conformado inicialmente por quince países. Este comité se reúne cada año con el motivo de de servir como órgano consultivo en los procesos de retiro y agregación a la lista del patrimonio mundial, monitoreo, presentación de informes entre otros puntos.

Teniendo en cuenta lo anterior, es necesario describir que la necesidad de sociedad influye en todas las facetas del ser humano ya que ``los aspectos culturales y naturales que revisten a todas las civilizaciones son fundamentales para entender sus prácticas cotidianas y festivas, presentes y pasadas. Estos aspectos se emplazan en el territorio, dibujando un entramado civilizatorio que distingue a cada tiempo y a cada sociedad de los demás''\cite*[p. 1]{rodriguezestudio}

Este tipo de patrimonio es algo a lo que no se le ha dado la importancia que se merece, especialmente a la documentación y estudio de la cuestión vernácula de la vivienda, el cual, debido a su importancia como lo menciona el \cite{icomos1999carta}, es una ``expresión fundamental de la identidad de una comunidad, de sus relaciones con el territorio y al mismo tiempo, la expresión de la diversidad cultural del mundo''.

Otro tema por tratar relacionado con los sistemas constructivos tradicionales es el valor inmaterial ya que este forma parte importante del patrimonio cultural pero el origen de estos ``es difícil rastrear ya que en parte es el resultado de siglos durante los cuales se daba por hecho que las sociedades heredan objetos del pasado que es necesario conservar, sin que esto obligara a una reflexión profunda de los valores y razones para su conservación'' \citep[p. 6]{villasenor2011valor}.

Unido a lo anterior, el valor inmaterial, como todo valor postulado hacia cualquier cosa, entidad, o pensamiento, es algo propuesto y generado por la sociedad, visto muchas veces fuera de los ojos de las personas que habitan o desarrollan el valor de las cosas.

Los elementos arquitectónicos y sistemas constructivos tradicionales en comunidades rurales, al formar parte de su historia, registro viviente de los cambios sociales, culturales, intimidad de las personas, formándose alrededor de diferentes creencias, costumbres y tradiciones se ha transformado en algo característico del ser humano por lo que forma parte de su patrimonio cultural, pero debido a su decadencia y desuso, está pasando a formar solo parte de la historia.

Dichas transformaciones no solo han afectado las formas de habitar, arquitectónicas y sociales, sino también han afectado al paisaje, cultura y tradiciones, por lo que es importante tener documentado dicho proceso, tanto social, constructivo, y teniendo en cuenta que: 

\begin{center}
    \begin{minipage}{0.9\linewidth}
        \vspace{4pt}%margen superior de minipage
        {\small...los conocimientos tradicionales presentan el inconveniente de que, por haber sido transferidos oralmente y mediante experiencias vivénciales de una generación a otra, rara vez se cuenta con documentos que permitan su caracterización y difusión. Además, como sucede con otras costumbres populares, es común que con el paso del tiempo vayan recibiendo influencias externas o alteraciones que en ocasiones acaban por desvirtuar sus bases originales.
        }
        \begin{flushright}
            \citep[p. 182]{guerrero2007arquitectura}
        \end{flushright}
        \vspace{4pt}%margen inferior de la minipage
    \end{minipage}
\end{center}

Junto con pasar de ser sensaciones que forman parte del patrimonio inmaterial, que se desarrollan hasta lo práctico, a lecturas donde no se puede transmitir lo mismo que la vivencia y experiencia misma del habitar en un hogar tradicional, en lugar de una búsqueda que implique retomar sus usos como una alternativa sustentable y competente.

Los avances tecnológicos, globalización e industrialización junto con diferentes aspectos que han surgido durante el tiempo en México así como los cambios culturales que estas han traído fueron causando de forma gradual que las comunidades originarias fueran desplazadas junto con sus saberes y tradiciones a algo marginal y que las grandes ciudades solo conocen como algo de lo que tienen una cierta clase de orgullo debido a los medios, pero desconocen en profundidad los aspectos que estas comunidades viven en su día a día y los cambios que están causando la pérdida del patrimonio tanto inmaterial como material que vienen de estas comunidades. 

El valle del Mezquital no se queda atrás en este aspecto, ya que mucha gente conoce la región por su producción de pulque, maguey y su cultura gastronómica, que fomentan el turismo tanto interno como externo del país, pero este turismo, crea una imagen ilusoria que, en algunos casos, genera una visión diferente de lo que se consideraría el vivir bien y vivir mal causando que las personas internas de las comunidades, salgan en busca de ese vivir bien, provocando la migración de sus miembros, específicamente hablando, ``la mayor parte de jóvenes en el Valle del Mezquital buscan migrar a Estados Unidos en busca de mejores ingresos que puedan mandar a sus familias para sustentar mejor sus gastos'' \citep{baez2012pueblos}.

Junto con todos estos cambios surge una contradicción entre la nueva tipología de vivienda y la tradicional, en la que se enfrentan estos dos tipos de conocimiento y nuevas ideas, pero esto puede suponer algo bueno, ya que una contradicción siempre ayuda a la evolución de alguna de las partes y ``en la sociedad humana, al igual que en la naturaleza, un todo único invariablemente se divide en diferentes partes; sólo hay diferencias en el contenido y la forma bajo condiciones concretas diversas. Siempre existirán opuestos como lo correcto y lo erróneo, lo bueno y lo malo, lo hermoso y lo feo. Solo cuando hay diferenciación y lucha, puede haber evolución. La verdad se desarrolla a través de su lucha con la falsedad'' \citep{mao1974cinco}.

Pero esto no quiere decir que la visión de un tipo de arquitectura se vuelva unilateral, sino que se puedan distinguir las ventajas y desventajas que  cada sistema constructivo aporta al contexto inmediato donde se planea implementar y con esto, poder llegar a la mejor decisión de construcción, la mas viable y con esto no generar cambios radicales en la forma de habitar de las personas que vayan a ocupar el inmueble.
