\section{Metodología}

En este apartado se describirán las herramientas y métodos que se utilizarán para la realización del proyecto de investigación, cada uno se dividirá por una propuesta de capítulos que se incluirán en el documento final.
% Metodología:

% \begin{enumerate}
%   \item La metodología es parte del proceso de investigación o método científico, que sigue la propedéutica, y permite sistematizar los métodos y técnicas necesarias para llevarla a cabo.
%   \item Instrumento que enlaza el sujeto con el objeto de investigación.
%   \item Guía y orienta la investigación.
%   \item Fija las normas de los objetos de investigación.
%   \item Procedimiento sistemático de la investigación.
% \end{enumerate}
%
% Método:
% \begin{enumerate}
%   \item El método es una forma de hacer un trabajo de investigación mas fácil.
%   \item Los métodos elegidos por el investigador facilitan el descubrimiento de conocimientos seguros y confiables que, potencialmente, solucionaran los problemas planteados en el proyecto de investigación.
%   \item Es el camino o medio para llegar a un fin, el modo de obrar y de proceder para alcanzar un objetivo determinado.
% \end{enumerate}

\subsection{Propuesta de capítulos}
\begin{enumerate}
  \item Introducción

    La introducción estará compuesta por los siguientes puntos:
  \begin{itemize}
    \item Tema a tratar
    \item Datos necesarios para que el lector comprenda el tema a tratar y lo que se esta proponiendo con el trabajo por lo tanto, se retomara como herramienta el protocolo de investigación de donde se obtendrá el objetivo general, justificación, planteamiento del problema, hipótesis etc. Redactado todo como un conjunto.
    \item Descripción breve de cada capítulo indicando que contiene cada uno.
    \item planteamiento de la idea principal del documento.

      \textbf{Método:}
    \item Síntesis

      \textbf{Herramientas:}
      \item Protocolo de investigación

  \end{itemize} 
  \item Documentación, análisis diacrónico de la vivienda tradicional y su relación con las construcciones industrializadas.
  % \item Análisis histórico de la vivienda tradicional

    La documentación arquitectónica se desenvolverá en tres factores conocidos como forma, función y tecnología. El análisis diacrónico se propone para conocer el estado actual de las viviendas existentes y su interacción con otro tipo de construcciones como lo son las industrializadas.

    \begin{itemize}
      \item Forma: Corresponde a la distribución de la vivienda en el solar, el número de cuartos, la geometría y el tamaño de los espacios.
      \item Función: El uso de los espacios y la forma en que se desarrolla la vida cotidiana de las personas que habitan la vivienda.
      \item Tecnología: Corresponde al tipo de materiales, sus métodos constructivos y las soluciones presentadas a problemas estructurales o de uso.

      \textbf{Subcapítulos:}
      \item Documentación arquitectónica

        \textbf{Método:}
        \begin{itemize}
          \item Síntesis
          \item Análisis
        \end{itemize}

      \item Análisis diacrónico de la vivienda tradicional

        \textbf{Método:}
        \begin{itemize}
          \item diacrónico
          \item Historico-comparativo
          \item Síntesis
          \item Análisis
        \end{itemize}

      \item Relación tradicional-industrializado

        \textbf{Método:}
        \begin{itemize}
          \item Análisis
          \item Síntesis
        \end{itemize}

      \textbf{Herramientas:}
      \item Levantamientos arquitectónicos.
      \item Modelado 3D de las viviendas tradicionales para tener un mejor entendimiento de la espacialidad, materiales y lograr una manera mas didáctica de visualización
      \item Fichas arquitectónicas que describan la localización, clima, materiales, espacialidad y descripciones sobre la bioconstrucción (Estas fichas son retomadas de los trabajos realizados por el Instituto Politécnico Nacional)
    \end{itemize}

  \item La incidencia de la migración en las transformaciones arquitectónicas.

    Este tipo de datos necesita una comparación, entre los datos recopilados mediante las entrevistas a profundidad que se realizaron a las personas de la comunidad. El objetivo es tener relatos vivenciales de la forma de habitar las construcciones tradicionales en la antigüedad y el modelo actual de vivienda que se encuentra en convivencia con las construcciones industrializadas.

    \textbf{Subcapítulos:}
    \begin{itemize}
      \item Incidencia arquitectónica
        
        \textbf{Método:}
        \begin{itemize}
          \item Cuantitativo
          \item Análisis
          \item Síntesis
        \end{itemize}

      \item Incidencia cultural y social
        
        \textbf{Método:}
        \begin{itemize}
          \item Cualitativo
          \item Análisis
          \item Síntesis
        \end{itemize}

      \textbf{Herramientas:}
      
      \item Entrevistas a profundidad (Taylor y Bogdan(1992, pp. 100-132) ''Introducción a los métodos cualitativos de investigación`` capítulo 4) que permitan conocer la historia y cambio de las formas de habitar de acuerdo a las vivencias personales de los habitantes.
      \item Fotografías actuales de los cambios originados en cuanto a los materiales en las construcciones tradicionales.
      \item Descripciones personales de las viviendas por parte de las personas que las habitan, para conocer lo que estas construcciones representan en cuanto a sus significados sociales y culturales.

    \end{itemize}

  \item Resignificación de la vivienda tradicional

    El objetivo de este capítulo es conocer la forma en que las personas que habitan las viviendas tradicionales las han ido resignificando a lo largo del tiempo, para conocer la forma en que se han ido adaptando a los cambios que se han presentado en la comunidad.

    \textbf{Subcapítulos:}
    \begin{itemize}
      \item Resignificación arquitectónica
        
        \textbf{Método:}
        \begin{itemize}
          \item Cuantitativo
          \item Análisis
          \item Síntesis
        \end{itemize}

      \item Resignificación cultural y social
        
        \textbf{Método:}
        \begin{itemize}
          \item Cualitativo
          \item Análisis
          \item Síntesis
        \end{itemize}

      \textbf{Herramientas:}
      \item Encuestas que permitan conocer el significado de la vivienda, contrastando familias que aún cuentan con conocimientos y viviendas tradicionales, con otras que ya no cuenten con estos aspectos.
      \item Relatos que permitan entender los cambios ideológicos que produzcan el anhelo de conservación o separación de la cultura y tradición.
      \item Diagramas que ayuden a entender gráficamente los procesos de cambio en la mentalidad de las personas en direcciones en pro y contra de los términos ''tradición`` y ''Actualidad``
    \end{itemize}

  \item Conclusiones
      \begin{itemize}
        \item Recapitular lo más importante de cada capítulo.
        \item Retomar la idea principal del proyecto.
        \item Propuestas sobre la solución del problema en caso de encontrar factibilidad para ello.
        \item Si las soluciones o el problema se pronostican en continuidad, argumentar los puntos personales hacia ese futuro
      \end{itemize}

\end{enumerate}
