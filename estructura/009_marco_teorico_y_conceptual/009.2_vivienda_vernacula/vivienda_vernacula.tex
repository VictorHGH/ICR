\subsection{Vivienda}
La vivienda es un elemento clave para la vida de las personas al fungir como un espacio donde se puede descansar y protegerse de las inclemencias del clima, así como también un lugar para desarrollar actividades cotidianas como cocinar, dormir y relacionarse con los demás. De igual manera es un requisito básico para la supervivencia así como un elemento importante en la cultura y la identidad de una sociedad. Cada cultura y época histórica ha desarrollado diferentes estilos y tipos de vivienda que reflejan sus valores, tradiciones y formas de vida.

La vivienda ha sido descrita y catalogada de diferentes maneras. Desde la vivienda rural, urbana, moderna, autoproducida, etc. pero siempre tendiendo en cuenta el factor humano, cultural e histórico arraigado en cualquiera de estas descripciones.

Instituciones como la \cite*{de1948declaracion} promueven a la vivienda como un derecho humano fundamental, por lo que se debe garantizar a todas las personas el acceso a una vivienda adecuada, segura y digna. De igual manera en el Pacto Internacional de Derechos Económicos, sociales y culturales \citep{humanos2012pacto} articulo 11° párrafo 1 se reconoce a la vivienda de un nivel adecuado como derecho para cualquier persona y con una mejora continua a dicho elemento.

En México, la constitución política de los Estados Unidos Mexicanos \citep{constitucion1917politica} en su artículo 4° párrafo 7 expresa el derecho de toda familia para disfrutar de una vivienda digna y decorosa.

Derivado de lo anterior, la ley de vivienda \citep{leyvivienda2019} en su artículo 2° define a la vivienda digna y decorosa como la ``que cumpla con las disposiciones jurídicas aplicables en materia de asentamientos humanos y construcción, salubridad, cuente con espacios habitables y auxiliares, así como con los servicios básicos y brinde a sus ocupantes seguridad jurídica en cuanto a su propiedad o legítima posesión, y contemple criterios para la prevención de desastres y la protección física de sus ocupantes ante los elementos naturales potencialmente agresivos''

Los elementos con los que debe contar una vivienda adecuada según la secretara de desarrollo agrario Territorial y Urbano \citep{reglas2021operacion} son los siguientes: seguridad de la tenencia, disponibilidad de servicios, materiales, instalaciones e infraestructura; que sea asequible; que disponga de buenas condiciones de habitabilidad y accesibilidad; así como que posea una ubicación que no comprometa la calidad de vida de sus habitantes y que su diseño responda a sus necesidades culturales.

La vivienda que cuenta con rezago habitacional, en contraste con la digna, es aquella en la ``que los materiales de techo, muros o pisos son considerados como precarios, que no cuentan con excusado o aquellas cuyos residentes habitan en hacinamiento'' \cite[p. 4]{reglas2021operacion}.

En este aspecto, el Consejo Nacional de Evaluación de la Política de Desarrollo Social \citep{rezagosocial2020} utiliza los siguientes indicadores para las viviendas con rezago habitacional: piso de tierra, con hacinamiento, sin disposición de excusado o sanitario, sin agua entubada de la red pública, sin drenaje y sin energía eléctrica.

En todo sentido, hay una diferenciación significativa entre la definición de vivienda digna y vivienda con rezago habitacional, ya que la primera se refiere a la calidad de la vivienda, mientras que la segunda se refiere a la calidad de vida de las personas que habitan en ella.

Si bien, las descripciones que se han hecho para la vivienda digna, van encaminadas al progreso de la sociedad y de sus habitantes, estas mismas segregan las particularidades que cada comunidad en el ámbito rural, pueden tener a sus nociones del habitar, su cultura y sus tradiciones.

CONEVAL pone al estado de Hidalgo en un grado de rezago social alto, ocupando el número 7 en el país \citep{indice2020rezago}, en este factor influye la vivienda con las mismas características que fueron descritas con anterioridad. Institudo Nacional de Estadística y Geografía \citep{censo2020poblacion} en el censo de población y vivienda 2020 coloca a Hidalgo con un total de viviendas particulares de 855,830 de las cuales 847,206. El municipio Cardonal se encuentra con un grado de rezago social bajo y en específico las comunidades de El Deca y El Buena cuentan con un indice de rezago bajo.

La comunidad del El Deca cuenta con una población total de 1,069 habitantes, un total de viviendas de 362 y siendo mas del 40\% de la población hablante de lengua indígena. Por otra parte, El Buena cuenta con 779 habitantes, 231 viviendas y al igual que El Deca cuenta con 40\% de hablantes de lengua indígena, en específico la lengua otomí o como ellos prefieren ser llamados: Hñähñus.

Estas dos comunidades forman parte del municipio del Cardonal, Hidalgo, México, el cual forma parte del ``Valle del Mezquital, el cual conforma una macroregión, compuesta por 27 municipios, que se caracteriza por un clima semidesértico, muy caliente durante el día y con bajas temperaturas por la noche. Hay una escasa precipitación y la vegetación es principalmente xerófila. La temperatura promedio de 13 grados centígrados y de 21 grados en los meses mas calurosos. La precipitación anual promedio es de 409 milímetros. Se clasifica la región en tres subregiones, con características de suelo diferentes, lo que hace que su población se relacione con el entorno de distinta manera`` \citep[p. 56]{alvarez2003maguey}.

Esta región y sus habitantes, forman parte importante de la cultura mexicana, ya que es la región donde se encuentra el maguey, planta que ha sido parte de la cultura mexicana desde tiempos prehispánicos, siendo parte de la alimentación, la medicina, la construcción y la vestimenta de los mexicanos. Al igual que hogar de uno de los grupos originarios mas importantes y conocidos de México, Los Otomíes.

``La historia prehispánica del grupo Otomí es muy oscura, creyéndose durante algún tiempo que tal grupo étnico es uno de loa más antiguos de Mesoaméricica`` \cite[p. 67]{GuerreroGuerrero1983}. Desde Cuicuilco y Copilco, hasta el Valle del Mezquital, los Otomies o mejor conocidos por ellos mismos como Hñähñus, son una de las culturas con mas enigma en su origen y desglose por desde el sur del Valle de México hasta el Valle del Mezquital pero no siendo estos los primeros habitantes del lugar.

El estudio a profundidad del origen de este grupo étnico se dificulta debido a la larga lista de grupos sociales, cambios que a sufrido la región y la perdida o falta de registros de los mismos a lo largo de la historia.

``A pesar de que el grupo otomí se asentó en la región árida del Valle del Mezquital, éste, por su ubicación y condiciones ecológicas, encerraba una variedad de hábitats y nichos. Su riqueza se hacía manifiesta en la diversidad de cactáceas, bosques de pino piñonero, agaves, yucas, mezquites, además de la nutrida fauna que frecuentaba sus chaparrales. Todo ello, combinado con el conocimiento ancestral adquirido por su contacto con los grupos nahuas, les abría la posibilidad de utilizarlo de manera variada, intensiva y acorde con los ciclos naturales.`` \citep{granados2004agricultura}

El Valle ha sido una región en constante movimiento desde sus inicios, desde los primeros asentamientos hasta lo que lo conforma hoy en día, pero siempre manteniendo los magníficos paisajes y su vegetación, siendo esto algo por lo que las personas de muchos lugares visitan los poblados de la región.

``La forma de habitar hoy en día se basa en el predio para cada familia, separada de las demás viviendas, a pesar de que las personas suelen conocerse por vínculos familiares o de amistad. Dejando lugar si el espacio lo permite, siempre espacio para el lugar de siembra''\cite*{GuerreroGuerrero1983}, esta forma de habitar se ve afectada en los principales municipios, donde la tipología arquitectónica a pasado a parecerse aún mas a la vista en las ciudades, en donde en conjunto con la tradición en el patrilocalismo donde el patriarca de la familia hereda a los hijos parte del terreno donde habitaron y las mujeres e hijo se mudan a dichos terrenos haciendo que los espacios cada vez se vean reducidos, provocando los cambios topológicos.

Es conocido que``la sociedad centromexicana durante el Posclásico Tardío tenía una estructura jerárquica. La familia era el núcleo básico. Los conjuntos de familias formaban una agrupación que los españoles llamaban la cuadrilla. Varias cuadrillas formaban un barrio. Los barrios se unían dentro de los señoríos. Algunos señoríos formaban confederaciones estratégicas.'' \citep[p. 156]{Otomies} hoy en día, la estructura familiar continua siendo hasta cierto punto, similar a la que se describe, ya que todavía se suele contar con un sistema patriarcal, en el que el hombre de la casa provee el sustento a toda la familia, y enseña a los hijos varones el oficio o da la oportunidad a que se desenvuelvan en el entorno laboral, mientras las mujeres se dedican al cuidado de la casa y a la enseñanza a las niñas del hogar.

Conformando parte de las tradiciones y la cultura Hñähñus, la arquitectura vernácula es parte importante de la identidad de los habitantes de estas comunidades, ya que es la forma en la que se ha construido la vivienda desde tiempos prehispánicos, y que ha sido transmitida de generación en generación, siendo parte de la cultura de estas comunidades.

\subsection{Vivienda vernácula}
Siendo los sistemas constructivos tradicionales parte de la historia de nuestro país, de las comunidades y su evolución a lo largo de los años así como de testigo de sus costumbres mas arraigadas, con una importancia tanto en su visión del mundo, la vivienda es parte fundamental en las relaciones personales, siendo el núcleo de desenvolvimiento de las personas, un lugar único para cada ser y familia, símbolo de protección y que en ocasiones, tiene un sentir propio, lleno de vida. ``Al entrar en la construcción de adobe, sienten que las acoge, que tiene vida, que tiene mucha historia de los abuelos, de sus generaciones pasadas''\citep{jeronimas}.

La arquitectura vernácula ``no responde a estilos, no representa épocas, no necesita de arquitectos, son quienes las habitan los encargados de modelarlas, ha estado allí, testigo de la cultura de los hombres`` \citep{gonzalez2017arquitectura}. Siendo parte del patrimonio de las comunidades y asentamientos humanos desde sus inicios, inventores de este tipo de arquitectura que no surge en busca de estética, proporción o cualquier otro sentir si no por la necesidad de un refugio, hecho con los materiales que cada localidad y comunidad tenia a su alcance, haciendo lo mejor que se podía conseguir con lo que se tenia a la mano.

``En el continente americano se ha encontrado evidencia de construcciones en tierra desde el periodo prehispánico. Con la llegada de los españoles se implementaron nuevos sistemas en los siglos XVI y XVIII empleaban sistemas como el bajareque -de tradición indígena-, la tapia y el adobe que, combinados con materiales como piedra y madera, se constituyen en la base material de la arquitectura que hoy hace parte del patrimonio cultural en nuestro continente"\citep{beltrantradicion}

``En los contextos arqueológicos, en muestras de adobes y tierra, se identifican especies como ahuehuete (Taxodium mucronatum), pino (Pinus sp.), sauce (Salix), ciprés (Cupressus), encino (Quercus), mezquite (Prosopis), maguey (Agave sp), nopal (Opuntia), huizache (Acacia), cardón (Ilex o Lemaireocereus), biznagas (Echinofossulocactus), yuca o palma (Yucca)... Muestra de la variedad de materiales con los que se solía construir en la región, siendo prueba de la capacidad de las poblaciones que habitaron la zona y su capacidad de adaptación a los climas secos y áridos del lugar.``\citep[p. 5]{aguilar2009}

Existen sistemas como el Adobe, bajareque, tapia, muros de piedra natural, paja, techos de materiales vegetales como la penca de maguey. Sistemas que cada comunidad adoptó y apropio, haciendo que estos fueran parte de sus conocimientos mas arraigados e importantes para cada uno, convirtiéndose en evidencia de su cultura y tradiciones.

La arquitectura vernácula, conformada por diferentes sistemas constructivos, dependientes de la región, clima, materiales locales y contexto sociocultural es el principio de toda arquitectura, desde que el ser humano paso a ser sedentario, conformando sistemas sociales más complejos y con esto, igualmente refugios más adecuados a su forma de vida. Las construcciones eran conformadas principalmente por tierra la cual es su componente principal, mezclada con algún aglomerante de fibras vegetales, materiales que se encontraban en el lugar donde se planeaba construir. Sistemas que fueron surgiendo como parte de la experimentación con los recursos que se tenían al alcance y que pasando de generación en generación estos conocimientos, fueron mejorando cada vez más y formando tradición en cada comunidad.

\citeauthor{rapoport1972vivienda} hace una de las primeras descripciones sobre la importancia de la vivienda en la cultura de la sociedad, desde la descripción de la tecnología y sus implicaciones en los cambios formales de las construcciones, hasta los usos que cada grupo es capaz de darle a los materiales y sistemas, creando conexiones profundas con la naturaleza o generando estatus relacionados con dichos conceptos.

Valeria prieto \citeyear{carrillo1978vivienda} nos describe como la ``vivienda campesina'' como la nombra ella, tiene una unión muy profunda con la naturaleza y el ser humano, ya que es en ella donde vivimos tristezas, felicidad, muerte, vida, etc. Pero tenemos que tener en cuenta que para nuestros días, este tipo de vivienda ha sido influenciada por la historia, refiriéndonos a todos los procesos sociales y económicos por los pasa cualquier sociedad a lo largo del tiempo.

\citeauthor{lopez1993arquitectura} nos describe algunas formas de acercamiento al conocimiento e historia de la arquitectura, desde los cronistas, murales y pinturas de la antigüedad debido a que se han encontrado vestigios en ruinas, documentos pictográficos, etc. De igual manera remarca la importancia de los materiales locales, forma, tecnología y función de la que fueron capaces nuestros antepasados.

Conociendo la importancia natural e histórica de los sistemas constructivos tradicionales también podemos encontrar a \citeauthor{zarate2009arquitectura} quien recalca la relación holística de la vivienda, aglomerando las creencias, cultura, naturaleza, religión y tiempo.

Sin embargo, este tipo de arquitectura a sido desplazada poco a poco debido a la marginación que la globalización junto con diferentes factores como la migración de las nuevas generaciones en las comunidades para la búsqueda de nuevas oportunidades, generan que los ideales de las personas al salir de su contexto sociocultural se mezcle con los de personas y ciudades, provocando que este quiera imitar lo que ve como nuevo y mejor en su lugar de origen, trayendo con esto un contexto muy diferente al que están acostumbrados, haciendo que las interacciones que se había entre la forma de habitar, desarrollarse y desenvolverse en su territorio, cambien radicalmente, incluso llegando a afectar sus costumbres y tradiciones, las cuales toman como algo antiguo y que pocas veces quieren mantener vivo en su forma de vida. 

La transición que la vivienda ha tenido a lo largo de los años en las diferentes regiones del Valle, son diferentes de acuerdo a la localidad, comunidad, nivel económico y social de las personas.

Hoy en día, los sistemas industrializados en la construcción, gracias a sus ventajas en comparación con los sistemas tradicionales, como lo pueden ser, frente a los factores ambientales, el no tener que preocuparse en si mismo por la construcción ya que una persona externa se encarga de realizarla. 

Llegando a volverse un identificador del nivel de vida de las personas en donde, en una sociedad, los sistemas tradicionales se llegan a considerar marginales o específicos de zonas rurales o lejanas a ciudades principales. Pero desde que se vieron algunos contras de los materiales industrializados como lo puede ser la contaminación y el deterioro de las zonas de donde se extraen enormes cantidades de materiales para realizarlos han resaltando los beneficios medioambientales que traen consigo los sistemas tradicionales, los cuales son sustentables, construidas con materiales naturales, con mantenimiento pueden perdurar muchos años. Pero como menciona \citep{gandara2000}, esta característica parece ser la única que está atrayendo al público hacia la importancia de este tipo de construcciones, incluso dejando de lado el valor cultural e histórico de su trasfondo.

La vivienda en su contexto actual se ha visto trasformada a través de cambios ideológicos y sociales provenientes de combinaciones entre grupos que migran fuera de sus lugares de origen. ``En el caso de la urbanización y la cada vez mayor tendencia a vivir en el mundo urbano influye en el deseo de las personas que no viven en las ciudades, de poseer una vivienda con referentes urbanos, aunque las formas físicas, organizativas, tecnológicos y usos no se correspondan con sus costumbres culturales'' \citep{alvarez2003maguey}.

Todo esto ha provocado la perdida y deterioro del patrimonio cultural construido así como el patrimonio cultural inmaterial de estas comunidades. Actualmente el patrimonio cultural de la región se caracteriza por su naturaleza y lugares de interés y conocidos por los lugareños, como se puede leer en el trabajo ``Estudio e identificación del patrimonio cultural y natural en el valle del mezquital''\citep{rodriguezestudio} el principal objeto en cualquier relación social-patrimonial son las personas las cuales mantienen estos lugares, conocimientos y tradiciones, ya que sin las personas que los visiten y valoren, estos se abandonarían y eventualmente se perderían.

Hace falta recalcar que en su estudio, no se encuentra el tipo de conocimiento inmaterial, ya que las personas son consientes del patrimonio cultural material refiriendo a lugares específicos, pero en ningún momento se habla del patrimonio inmaterial implícito en sus tradiciones, fiestas y en este caso, los sistemas constructivos tradicionales, cuyo conocimiento pasa de manera generacional a los más jóvenes.
