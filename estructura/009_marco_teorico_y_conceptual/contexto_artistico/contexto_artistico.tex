\subsection{Contexto artístico}
\subsubsection{Patrimonio cultural}
El patrimonio cultual, ``entendido como todo aquello que socialmente se considera digno de conservación independientemente de su interés utilitario" \citep{prats1998concepto}, a sido estudiado de diversas maneras, desde el nivel profundo de la definición de cada uno de sus conceptos por separado hasta el significado de importancia global al cual se le acuña hoy en día, los alcances del patrimonio van desde los ámbitos cultural y natural, siendo dirigido a elementos tanto materiales como inmateriales. Los elementos arquitectónicos y sistemas constructivos tradicionales en comunidades rurales, al formar parte de su historia, registro viviente de los cambios sociales, culturales, testigo de la intimidad de las personas, formándose alrededor de diferentes creencias, costumbres y tradiciones se ha transformado en algo característico del ser humano por lo que forma parte de su patrimonio cultural, pero debido a su decadencia y desuso, está pasando a formar solo parte de la musealización y documentación, junto con pasar de ser sensaciones que forman parte del patrimonio inmaterial, que se desarrollan hasta lo práctico, a lecturas donde no se puede transmitir lo mismo que la vivencia y experiencia misma del habitar en un hogar tradicional, en lugar de una búsqueda que implique retomar sus usos como una alternativa sustentable y competente.

El valor intrínseco cuyo origen ``es difícil rastrear ya que en parte es el resultado de siglos durante los cuales se daba por hecho que las sociedades heredan objetos del pasado que es necesario conservar, sin que esto obligara a una reflexión profunda de los valores y razones para su conservación"\citep{villasenor2011valor}. Pero como todo valor postulado hacia cualquier cosa, entidad, o pensamiento, es algo propuesto y generado por la sociedad, visto muchas veces desde el exterior incluso fuera de los ojos de las personas que habitan o desarrollan el valor intrínseco de las cosas. Siendo esto parte de la construcción social del patrimonio. El valor patrimonial, ha llegado a ser búsqueda de instituciones privadas, estatales y gubernamentales que buscan beneficiarse de el, haciendo que el término patrimonio, no pase a ser mas que algo monetario.

La -Convención para la salvaguardia del patrimonio cultural inmaterial en 2003-\citep{UNESCO2003} y la -Convención sobre la protección del patrimonio mundial, cultural y natural de 1972- realizadas por la Organización de las Naciones Unidas para la Educación, la Ciencia y la Cultura, han sido puntos claves y de inflexión en el tema del patrimonio cultural e inmaterial, dando pautas, estudiando los problemas, generando políticas y la unión de los países participantes en la búsqueda de la protección del patrimonio Cultural, en 1972 creando el ``Comité del Patrimonio Mundial intergubernamental'' conformado primeramente por quince países. Este comité se reúne cada año con el motivo de de servir como órgano consultivo.

``Los aspectos culturales y naturales que revistes a todas las civilizaciones son fundamentales para entender sus prácticas cotidianas y festivas, presentes y pasadas. Estos aspectos se emplazan en el territorio, dibujando un entramado civilizatorio que distingue a cada tiempo y a cada sociedad de los demás"\citep{rodriguezestudio}. Estos aspectos se han visto afectados por el crecimiento y cambio rápido de las sociedades por la globalización, avances tecnológicos y el crecimiento acelerado de la población, especialmente en los países en desarrollo.

\subsubsection{Documentación}
Hoy en día, la documentación del patrimonio cultural es algo a lo que no se le ha dado la importancia que se merece, especialmente a la documentación del patrimonio vernáculo, el cual, debido a su importancia como ``expresión fundamental de la identidad de una comunidad, de sus relaciones con el territorio y al mismo tiempo, la expresión de la diversidad cultural del mundo''\citep{icomos1999carta} tiene para toda comunidad. La idea de su documentación y catalogación viene fundamentada a que cada localidad, como se mencionó anteriormente, tiene factores diferentes en los ámbitos: ambientales, sociales y culturales, ya que cada grupo social fue generando y transformando dichos sistemas vernáculos a su forma de habitar, necesidades, territorio y materiales a su alcance llegando ser una muestra de la capacidad creativa de la raza humana, mostrar los valores humanos y culturales de una determinada época de la historia, ser testigo de la cultura de las comunidades, ejemplificando un tipo de arquitectura que a pesar de ser diversa en sus métodos constructivos, tienen la misma base fundamental, basada en tierra y con materiales locales, ejemplo del habitat de humanos con culturas pasadas y está asociada a tradiciones, costumbres y conocimientos vivos desde la antigüedad. Siendo estos últimos requisitos necesarios para logar catalogar algo como patrimonio mundial \citep{WEB_requisitos_patrimonio_mundial}.

Estos requisitos a pasear de ser para patrimonio mundial, tienen características que forman parte de lo que son los sistemas constructivos tradicionales, teniendo únicamente como punto diferente y siendo parte del patrimonio inmaterial, la relación entre las diferentes ideologías y sus viviendas, las cuales se rodean de misticismo, supersticiones y creencias.

Como parte de la documentación del patrimonio cultural, se toman como ejemplo las directrices y previo análisis sobre la el tema, abarcando resultados desde otros países hasta la ciudad de cuenca, sintetizando información referente a diferentes instituciones y sus metodologías para este tipo de documentación\citep{narvaez_metodologipara}.

La ayuda de este tipo de documentación con énfasis en recolectar las relaciones entre lo material, inmaterial, historia y cultura, buscan creas bases fuertes que fundamenten el cuidado, preservación y revalorización todo lo que rodea al patrimonio tradicional arquitectónico.

