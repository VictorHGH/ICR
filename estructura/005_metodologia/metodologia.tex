\section{Metodología}

Este capítulo detalla el diseño cualitativo-participativo que permitirá
poner a prueba la hipótesis expuesta en la Introducción y operacionalizar
la matriz presentada en la sección \ref{subsec:sintesis_operacional} del
marco teórico-conceptual.

%--------------------------------------------------------------------
\subsection{Diseño general de investigación}
%--------------------------------------------------------------------

\begin{itemize}
	\item \textbf{Enfoque:} estudio de caso múltiple con lógica etnográfica
	      y de historia de vida de la vivienda.
	\item \textbf{Ámbito espacial:} comunidades ``El Deca'' y ``El Buena'',
	      municipio de El Cardonal, Hidalgo.
	\item \textbf{Horizonte temporal:} periodo 1990-2024, coincidente con
	      el lapso de acelerada sustitución de técnicas tradicionales.
	\item \textbf{Muestra:} \emph{n} = 8 viviendas
	      (4 por comunidad), seleccionadas por muestreo
	      intencional heterogéneo para cubrir rangos de antigüedad,
	      estado de conservación y ocupación.
\end{itemize}

%--------------------------------------------------------------------
\subsection{Estrategias de recolección de datos}
%--------------------------------------------------------------------

\vspace{-0.5em}
\begin{enumerate}
	\item \textbf{Censo arquitectónico diacrónico}\\
	      \textit{Instrumentos:} fichas de levantamiento, fotografías, vídeos.\\
	      \textit{Producto:} planos CAD y documentación digital.
	\item \textbf{Entrevistas semiestructuradas}\\
	      \textit{Sujetos:} propietarios (n = 7), maestros albañiles (n = 3). \\
	      \textit{Eje temático:} transmisión del saber, decisiones de
	      sustitución, percepción de valores patrimoniales.
	\item \textbf{Observación participante}\\
	      Registro audiovisual de faenas comunales y rituales asociados a la construcción.
	\item \textbf{Encuesta socio-económica breve}\\
	      Reúne datos sobre ingresos, destino de remesas, gasto en materiales y mano de obra.
	\item \textbf{Talleres de cartografía participativa}\\
	      Los habitantes localizan en mapas SIG los recursos del
	      sistema constructivo (bancos de arcilla, magueyales, pozos de
	      agua) y valoran con adhesivos de color los niveles percibidos de
	      amenaza y relevancia cultural.
	\item \textbf{Creación de base de datos para el resguardo de la información}\\
	      Se crea mediante tecnologías de la información, una base de datos que permita
	      almacenar la información relevante sobre las construcciones analizadas.
\end{enumerate}

%--------------------------------------------------------------------
\subsection{Técnicas de análisis}
%--------------------------------------------------------------------

\begin{itemize}
	\item \textbf{Análisis tipológico} de fases constructivas mediante relatoría de los habitantes.
	\item \textbf{Codificación temática} de entrevistas en NVivo, guiada
	      por la matriz de valores (uso, simbólico, histórico, económico).
	\item \textbf{Matriz de ponderación} de factores de ruptura:
	      puntajes 0-3 asignados por triangulación de tres fuentes
	      (entrevista, encuesta, observación) y validados en
	      \textit{focus group}.
	\item \textbf{Índice de Vitalidad Técnica (IVT)} IVT =
	      (M1 + M2 + M3 + M4 + M5)/5, donde
	      M1 = maestros activos,
	      M2 = edad de ingreso,
	      M3 = faenas/año,
	      M4 = \% de cubiertas de penca,
	      M5 = \% de rituales vigentes.
	      Se categoriza en alta (\textgreater0.66), media (0.33-0.66),
	      baja (\textless0.33).
\end{itemize}

%--------------------------------------------------------------------
\subsection{Consideraciones éticas y de devolución}
%--------------------------------------------------------------------

\begin{enumerate}
	\item Consentimiento informado bilingüe (español/hñähñu).
	\item Anonimato codificado de entrevistados.
	\item Devolución de resultados mediante cuadernillo ilustrado
	      entregado en asamblea comunitaria y vía PDF.
\end{enumerate}

%--------------------------------------------------------------------
\subsection{Limitaciones del estudio}
%--------------------------------------------------------------------

\begin{itemize}
	\item \textbf{Representatividad geográfica} limitada a dos comunidades;
	      transferibilidad se discutirá comparando literatura de casos
	      análogos (Mixteca, Sierra Norte de Puebla).
	\item \textbf{Tiempo de campo} restringido a la posibilidad temporal del investigador;
	      algunos rituales estacionales podrían no ser observados en directo.
	\item \textbf{Sesgo de deseabilidad social} mitigado con observación
	      participante y triangulación de fuentes.
\end{itemize}

%--------------------------------------------------------------------
\subsection{Cronograma y productos}
%--------------------------------------------------------------------

Con esta metodología se busca vincular evidencia empírica y reflexión
crítica, aportando bases sólidas para la discusión de estrategias de
revitalización del saber constructivo hñähñu.
