% La migración que se ha dado en las regiones rurales durante el pasar de los años debido a las dificultades económicas han afectado de igual manera a las comunidades de \emph{El Deca} y \emph{El Buena} generando paisajes llenos de viviendas realizadas con materiales industrializados, provocando cambios ideológicos y socioculturales que han desplazado a los conocimientos tradicionales de una manera marginal, haciendo que sean vistos como algo que se quiera evitar o aprender.

% La vivienda vernácula ``Arquitectura sin pedigrí, Arquitectura no formal, arquitectura no clasificada'' \citep[p .9]{rudofsky1976arquitectura}

% ``Ante este panorama de desigualdad rural-urbana reflejada en la baja calidad de vida, falta de oportunidades de empleo, escasa rentabilidad de sus productos e incremento de los niveles de pobreza, los habitantes de los espacios rurales implementan estrategias en la búsqueda de mejores oportunidades socioeconómicas que les permitan mantener su producción agropecuaria y sus condiciones de vida; una de ellas es la migración laboral nacional o internacional'' \citep[p. 126]{monroy2009}

%"Neovivienda" "Trnasculturación" (Juarez Sánchez et al.,2018, p. 206-207 "transformación de la vivienda rural en México")

% A su vez, tanto los conocimientos constructivos como tradicionales se han visto afectados de igual manera y sumando a esto, ``antes del siglo XX no se habían considerado las edificaciones vernáculas como valores arquitectónicos, normalmente se manejaron conceptos elitistas, que menospreciaban a estas construcciones'' \citep[p. 2]{torrez1999revista}.

% En conjunto, el significado con el que se asociaba esta arquitectura y el cambio económico en donde, ``a partir de 1983 el sector agrícola inició la transición hacia un nuevo esquema de desarrollo caracterizado por una menor intervención del Estado en las actividades productivas, la apertura comercial externa, la desregulación de la economía y el equilibrio en las finanzas públicas'' \citep[p. 41]{calderon2008politica}, produjo una desigualdad rural-urbana que provoco que la necesidad económica, falta de oportunidades y pobreza pusiera en marcha la migración de un alto indice de la población mexicana, entre ellos, las comunidades del Valle del Mezquital.

% Dicha migración a dado como resultado la mezcla de ideologías y formas culturales, que sumado a que ``... los mexicanos que pasan allá largas temporadas van asimilando algunos modos de vida diferentes a los de sus lugares de origen''\citep[p. 28]{boils2010dadho}, han provocado transformaciones en la vivienda tradicional, inclusive llegando a la desaparición de la misma y su significado, el cual va mas allá de lo constructivo, acercándose a la cosmovisión y formas de habitar a las que estaba acostumbrada la población.

La arquitectura vernácula del Valle del Mezquital constituye una expresión viva del conocimiento tradicional y de la relación simbiótica entre los habitantes y su entorno natural. Sin embargo, en las últimas décadas, estos sistemas constructivos han experimentado un proceso acelerado de abandono y transformación, resultado de múltiples factores que incluyen la migración, la industrialización de materiales de construcción, los cambios socioeconómicos, la transculturación ideológica y la falta de reconocimiento institucional hacia el valor de la arquitectura vernácula.

La vivienda tradicional, definida como ``arquitectura sin pedigrí, arquitectura no formal, arquitectura no clasificada'' \citep[p. 9]{rudofsky1976arquitectura}, ha sido desplazada progresivamente en favor de edificaciones realizadas con materiales industrializados, alterando no solo el paisaje construido, sino también las formas de habitar y las cosmovisiones asociadas a dichas prácticas constructivas.

Aunque fenómenos como la migración laboral —producto de desigualdades rurales-urbanas, precarización del campo y apertura económica— han tenido un impacto importante \citep{monroy2009, boils2010dadho}, no constituyen la única causa. La percepción desvalorizada de las construcciones vernáculas, históricamente considerada de menor valor arquitectónico \citep[p. 2]{torrez1999revista}, y la falta de políticas públicas que promuevan su conservación, también han sido determinantes en este proceso de pérdida.

Este desplazamiento de los saberes constructivos tradicionales no solo implica la pérdida material de edificaciones, sino también la erosión de prácticas culturales, conocimientos ecológicos locales y modos de vida sostenibles que articulaban la identidad social de las comunidades del Valle del Mezquital.

Frente a este escenario, resulta fundamental analizar las dinámicas que han conducido a la desvalorización de los sistemas constructivos vernáculos y explorar alternativas que permitan su reconocimiento, revitalización y preservación como parte del patrimonio cultural biocultural de la región.
